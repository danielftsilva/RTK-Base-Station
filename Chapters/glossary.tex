%!TEX root = ../template.tex
%%%%%%%%%%%%%%%%%%%%%%%%%%%%%%%%%%%%%%%%%%%%%%%%%%%%%%%%%%%%%%%%%%%%
%% glossary.tex
%% NOVA thesis document file
%%
%% Glossary definition
%%%%%%%%%%%%%%%%%%%%%%%%%%%%%%%%%%%%%%%%%%%%%%%%%%%%%%%%%%%%%%%%%%%%

\typeout{NT FILE glossary.tex}

% Glossary - Lista de palavras que explica termos obscuros por meio de outros conhecidos.

\newglossaryentry{computer}{
	name={computer}, 
	description={An electronic device which is capable of receiving information (data) in a particular form and of performing a sequence of operations in accordance with a predetermined but variable set of procedural instructions (program) to produce a result in the form of information or signals. It is also possible to add citations to the glossary!}
}
\newglossaryentry{base_station}{
	name={base station},
	description={A device used to receive and send orders from a sender to a (mobile) receiver. It is also possible to add citations to the glossary!}
}

\newglossaryentry{ephemeris}{
	name={ephemeris},
	description={Chronical list of data that represents the orbital path of an object over time~\cite{novatel_gnss}.}
}
\newglossaryentry{geodetic_datum}{
	name={geodetic datum},
	description={A reference surface or base on which accurate positioning measurements are made. Essentially, it identifies the origin on a certain measurement scale~\cite{datum_2013}.}
}

\newglossaryentry{pseudorandom}{
	name={pseudorandom},
	description={Relative to the generation of random numbers through a explicit computational operation, ir order to satisfy a statistical test.}
}
\newglossaryentry{correlation}{
	name={correlation},
	description={Statistical representation of the level of tendency of mutual variation between two or more attributes or measures belonging to the same group.}
}

\newglossaryentry{geoid}{
	name={geoid},
	description={The irregular shape that the ocean surface would take, considering the gravitational and rotational influences of the Earth, in the absence of winds and tides.}
}

%This is a test that adds a citation \cite{Artho04} to the glossary!
