%!TEX root = ../template.tex
%%%%%%%%%%%%%%%%%%%%%%%%%%%%%%%%%%%%%%%%%%%%%%%%%%%%%%%%%%%%%%%%%%%%
%% chapter3.tex
%% NOVA thesis document file
%%
%% Chapter with a short latex tutorial and examples
%%%%%%%%%%%%%%%%%%%%%%%%%%%%%%%%%%%%%%%%%%%%%%%%%%%%%%%%%%%%%%%%%%%%

\typeout{NT FILE chapter3.tex}
% se eu quiser mais um capitulo, tenho de copiar estas linhas
% (da 1-9) -> e com as devidas alterações (ex.: chapter5, acknowledgements 2)

\chapter{Workflow Planning}\label{cha:III_workflow}

With the state of the art of precise navigation addressed, the purpose of this chapter is to clarify the expected workflow of the project to develop through this Dissertation. For that, Table~\ref{tab:workflow} represents the entire division of the process into six major phases, which are composed of tasks.

Addressed in this document are the first two phases, represented in Chapter~\ref{cha:II_SotA} and Section~\ref{sec:III_requirements}, respectively. These parts make up the theoretical component of the project, which naturally opens the way for the practical component, guided by phases 3 to 6.

Starting with the design of the device's hardware, phase 3 aims to effectively lower the entire system's power consumption, which will be started by first rearranging the blocks of its architecture -- represented earlier by Figure~\ref{fig:architecture_original} --, followed by the design of the electrical circuit and respective Printed Circuit Board (PCB) after selecting the functionality-defining components.

To conclude the electrical aspect of the project, a prototyping phase (phase 4) is expected. This phase will provide a foundation for the firmware design phase (phase 5), which, based on the system's interconnections between units and modules will grant a stable operation of the device.

Finally, phase 6 will help solidify all the work developed in the previous phases through the evaluation of the device's performance with respect to the desired functionalities (i.e. the requirements).

\section{System Requirements}\label{sec:III_requirements}

Aiming to implement a low-power solution for a new, improved version of the beRTK\textsuperscript{\textregistered} Base Station, the next step to take after gathering all the necessary knowledge on the subjects is to effectively define the system requirements -- which corresponds to phase 2 in the expected workflow\footnote{Phases 1 and 2 are backed by the writing of a Dissertation Preparation report, which corresponds to this very document; The full Dissertation report will contain all the expected project workflow phases.} (Table~\ref{tab:workflow}).

For the development of a project, Beyond Vision's way of establishing its specifications is through a grammar table defined by codes in the following manner:

\begin{center}
	\textbf{<PROJ>.<SUBPROJ>.<SUBJECT>.<SERIAL>}
\end{center}
Where:
\begin{itemize}
	\item \textbf{<PROJ>}: Identifies the project. In this case it will be referred to as RTKBS;
	\item \textbf{<SUBPROJ>}: Identifies the subproject related to the specification;
	\item \textbf{<SUBJECT>}: Subgroups the specifications by area;
	\item \textbf{<SERIAL>}: Completes the individual specification with a unique ID number.
\end{itemize}

Therefore, the main requirements gathered for the project to develop are presented in Tables~\ref{tab:CRT_requirements}-\ref{tab:SW_requirements}, in the following sections.

\begingroup
\begin{table}[h!]
	\captionsetup{justification=centering}
    \caption{Expected project workflow.}
	\label{tab:workflow}
	\centering%@{}l@{}@{}c@{}@{}c@{}@{}c@{}@{}c@{}
    % \setlength{\tabcolsep}{10pt} % Default value: 6pt
    % \renewcommand{\arraystretch}{1.5} % Default value: 1
	\begin{tabular}{ll}
        \toprule
        \textbf{Phase} & \textbf{Observations} \\
        \midrule
        \multirow{2}*{\textbf{1. Previous Studies}} & Includes state of the art analysis, selection of typologies, \\
                            						& technologies, etc. \\
		\midrule
		\textbf{2. Specifications} 					& Survey of functional and non-functional specifications \\
		\midrule
		\textbf{3. Hardware Design} 				& \\
		\multirow{2}*{3.1. Architecture}			& System architecture design, including interconnection of \\
													& modules, functionalities, etc. \\
		\multirow{2}*{3.2. Component Selection}		& Selection of major circuit components (those that define \\
													& functionality) \\
		3.3. Circuit Design							& Electrical circuit design \\
		3.4. PCB Design								& Design of the PCB(s) needed \\
		\midrule
		\textbf{4. Prototyping} 					& \\
		4.1. PCB Manufacturing						& Shipment of PCB(s) for manufacturing \\
		4.2. Component Purchase						& Organization and purchase of the components involved \\
		4.3. Assembly								& Final prototype assembly (PCB(s), cable(s), etc.) \\
		\midrule
		\textbf{5. Firmware Design} 				& \\
		5.1. Architecture							& FW architecture design: modules, interconnections, etc. \\
		\multirow{2}*{5.2. Component Selection}		& Selection of libraries and other applicable FW \\
													& components \\
		5.3. Coding									& Code development applicable to the project \\
		\midrule
		\textbf{6. Testing} 						& \\
		6.1. Bench testing 							& Confirmation of the intended functionalities \\
		6.2. Field testing							& Confirmation of the intended functionalities \\
		\multirow{2}*{6.3. Performance testing}		& Survey of the system's limitations (range, crosstalk, \\
													& accuracy, autonomy, etc.) \\
		\multirow{2}*{6.4. Certification testing}	& If applicable: certification tests for the current \\
													& applicable European standards \\
        \bottomrule
    \end{tabular}
\end{table}
\endgroup

\begingroup
\begin{table}[h!]
	\captionsetup{justification=centering}
    \caption{Gantt chart.}
	\label{tab:workflow_timeline}
	\centering
	\begin{tabular}{c|p{0.6cm}p{0.6cm}p{0.6cm}p{0.6cm}p{0.6cm}p{0.6cm}p{0.6cm}p{0.6cm}p{0.6cm}p{0.6cm}p{0.6cm}p{0.6cm}}
    	\multicolumn{1}{c}{} & Oct & Nov & Dec & Jan & Feb & Mar & Apr & May & Jun & Jul & Aug & Sep \\ \hline
		\multirow{2}*{Phase 1} 			& \cellcolor{blue!25} & \cellcolor{blue!25} & \cellcolor{blue!25} & \cellcolor{blue!25} & \cellcolor{blue!25} & & & & & & & \\ 
										& \cellcolor{blue!25} & \cellcolor{blue!25} & \cellcolor{blue!25} & \cellcolor{blue!25} & \cellcolor{blue!25} & & & & & & & \\ \hline
		\multirow{2}*{Phase 2} 			& \cellcolor{blue!25} & \cellcolor{blue!25} & \cellcolor{blue!25} & \cellcolor{blue!25} & \cellcolor{blue!25} & & & & & & & \\ 
										& \cellcolor{blue!25} & \cellcolor{blue!25} & \cellcolor{blue!25} & \cellcolor{blue!25} & \cellcolor{blue!25} & & & & & & & \\ \hline
		\multirow{2}*{Phase 3} 			& & & & & & \cellcolor{blue!25} & \cellcolor{blue!25} & & & & & \\ 
										& & & & & & \cellcolor{blue!25} & \cellcolor{blue!25} & & & & & \\ \hline
		\multirow{2}*{Phase 4} 			& & & & & & & & \cellcolor{blue!25} & & & & \\ 
										& & & & & & & & \cellcolor{blue!25} & & & & \\ \hline
		\multirow{2}*{Phase 5} 			& & & & & & \cellcolor{blue!25} & \cellcolor{blue!25} & \cellcolor{blue!25} & & & & \\ 
										& & & & & & \cellcolor{blue!25} & \cellcolor{blue!25} & \cellcolor{blue!25} & & & & \\ \hline
		\multirow{2}*{Phase 6} 			& & & & & & & & & \cellcolor{blue!25} & \cellcolor{blue!25} & \cellcolor{blue!25} & \\ 
										& & & & & & & & & \cellcolor{blue!25} & \cellcolor{blue!25} & \cellcolor{blue!25} & \\ \hline
		\multirow{2}*{}Dissertation		& \cellcolor{blue!25} & \cellcolor{blue!25} & \cellcolor{blue!25} & \cellcolor{blue!25} & \cellcolor{blue!25} & & & & & & & \\
					   Prep. Report 	& \cellcolor{blue!25} & \cellcolor{blue!25} & \cellcolor{blue!25} & \cellcolor{blue!25} & \cellcolor{blue!25} & & & & & & & \\ \hline
		\multirow{2}*{}Dissertation 	& \cellcolor{blue!25} & \cellcolor{blue!25} & \cellcolor{blue!25} & \cellcolor{blue!25} & \cellcolor{blue!25} & \cellcolor{blue!25} & \cellcolor{blue!25} & \cellcolor{blue!25} & \cellcolor{blue!25} & \cellcolor{blue!25} & \cellcolor{blue!25} & \cellcolor{blue!25} \\ 
							 Report  	& \cellcolor{blue!25} & \cellcolor{blue!25} & \cellcolor{blue!25} & \cellcolor{blue!25} & \cellcolor{blue!25} & \cellcolor{blue!25} & \cellcolor{blue!25} & \cellcolor{blue!25} & \cellcolor{blue!25} & \cellcolor{blue!25} & \cellcolor{blue!25} & \cellcolor{blue!25} \\ \hline				 
    \end{tabular}		
\end{table}
\endgroup

\subsection{Certification Requirements}\label{III:CRT_requirements}

This section describes the main certification requirements of the device.

\begingroup
\begin{table}[H]
	\captionsetup{justification=centering}
    \caption{beRTK\textsuperscript{\textregistered} Base Station certification requirements.}
	\label{tab:CRT_requirements}
	\centering

	% mini tabela: 1, 2, 3
	\begin{tabular}{rl}
        \toprule
		\textbf{RTKBS.MAIN.CRT.010} 			& \textbf{European Regulatory Framework} \\
		\multirow{2}*{}							& The RTK Base Station shall respect the current European \\
												& regulatory framework. \\
		\midrule
		\multirow{1}*{\textbf{Remarks:}}   & \\
		\bottomrule
		&\\
		&\\
		\toprule
		\textbf{RTKBS.MAIN.CRT.020} 		& \textbf{Electromagnetic Compatibility Certification} \\
		\multirow{2}*{}						& The RTK Base Station shall respect the current European \\
											& EMC (Electromagnetic Compatibility) standards. \\
		\midrule
		\multirow{2}*{\textbf{Remarks:}} 	& \emph{It must follow the European norms ETSI EN301 489-1,} \\
							  				& \emph{EN301 489-3 and EN301489-17.}\\
		\bottomrule
		&\\
		&\\
        \toprule
		\textbf{RTKBS.MAIN.CRT.030} 		& \textbf{Radio Equipment Certification} \\
		\multirow{4}*{}						& The RTK Base Station shall respect the current European \\
											& Radio Equipment Directive (RED), either by running \\
											& tests or by ensuring that all the radio equipment \\
											& on-board is already certified by its manufacturer. \\
		\midrule
		\multirow{4}*{\textbf{Remarks:}} 	& \emph{The list of RED related standards to be respected is} \\
							  				& \emph{at least (but not restricted to) the following:} \\
											& \emph{EN300 328:2015 (Wi-Fi); EN300 440 (Wi-Fi, ZigBee,} \\
											& \emph{Bluetooth); EN303 413 (GPS, RTK).}\\
		\bottomrule
	\end{tabular}
\end{table}
\endgroup

\subsection{Environmental Requirements}\label{III:ENV_requirements}

This section describes the main environmental requirements of the device.

\begingroup
\begin{table}[H]
	\captionsetup{justification=centering}
    \caption{beRTK\textsuperscript{\textregistered} Base Station environmental requirements.}
	\label{tab:ENV_requirements}
	\centering

	% mini tabela: 4, 5
	\begin{tabular}{rl}
        \toprule
		\textbf{RTKBS.MAIN.ENV.010} 			& \textbf{Waste Management Certification} \\
		\multirow{3}*{}							& The RTK Base Station shall respect the current European \\
												& relating to WEEE (Waste of Electric and Electronic \\
												& Equipment). \\
		\midrule
		\multirow{3}*{\textbf{Remarks:}}   		& \emph{It must follow the following directives: 2012/19/EU} \\
												& \emph{(WEEE); 2006/66/EC (Battery recycling); 2011/65/EC} \\
												& \emph{(RoHS).} \\
		\bottomrule
		&\\
		&\\
        \toprule
		\textbf{RTKBS.MAIN.ENV.020} 		& \textbf{IP Protection} \\
		\multirow{2}*{}						& The RTK Base Station shall have an IP environmental \\
											& protection degree of IP65. \\
		\midrule
		\multirow{1}*{\textbf{Remarks:}} 	& \emph{Following European standard EN60529.} \\
		\bottomrule
	\end{tabular}
\end{table}
\endgroup

\subsection{Installation Requirements}\label{III:INST_requirements}

This section describes the main installation requirements of the device.

\begingroup
\begin{table}[H]
	\captionsetup{justification=centering}
    \caption{beRTK\textsuperscript{\textregistered} Base Station installation requirements.}
	\label{tab:INST_requirements}
	\centering

	% mini tabela: 6
	\begin{tabular}{rl}
        \toprule
		\textbf{RTKBS.MAIN.INST.010} 			& \textbf{Easiness of Field Deploy} \\
		\multirow{4}*{}							& It shall be easy and fast to deploy the RTK base \\
												& Station on the field. Namely, it should only be needed \\
												& to choose a place, place the module, and turn the \\
												& module on.\\
												
		\midrule
		\multirow{1}*{\textbf{Remarks:}}   		& \\
		\bottomrule
	\end{tabular}
\end{table}
\endgroup

\subsection{Functional Requirements}\label{III:FCT_requirements}

This section describes the main functional requirements of the device.

\begingroup
\begin{table}[H]
	\captionsetup{justification=centering}
    \caption{beRTK\textsuperscript{\textregistered} Base Station functional requirements.}
	\label{tab:FCT_requirements}
	\centering

	% mini tabela: 7, 8, 9, 10, 11
	\begin{tabular}{rl}
        \toprule
		\textbf{RTKBS.MAIN.FCT.010} 			& \textbf{RTK Base} \\
		\multirow{5}*{}							& As an RTK Base Station, the main purpose of this device \\
												& is to accurately calculate its position. It shall also \\
												& transmit the position wirelessly to the devices around \\
												& that are capable of receiving and processing this \\
												& information. \\
		\midrule
		\multirow{1}*{\textbf{Remarks:}}   & \\
		\bottomrule
		&\\
		&\\
		\toprule
		\textbf{RTKBS.MAIN.FCT.020} 		& \textbf{Positioning Error} \\
		\multirow{2}*{}						& The RTK Base Station shall achieve a positioning error \\
											& of less than 2 cm. \\
		\midrule
		\multirow{2}*{\textbf{Remarks:}} 	& \emph{The u-Blox module ZED-F9P should be the basis for} \\
							  				& \emph{this RTK system.}\\
		\bottomrule
		&\\
		&\\
        \toprule
		\textbf{RTKBS.MAIN.FCT.030} 		& \textbf{On-board Intelligence} \\
		\multirow{4}*{}						& The RTK Base Station shall have an MCU, able to carry \\
											& out a pre-determined set of instructions, accept \\
											& instructions in real-time, receive NTRIP information \\
											& and send correction signals to the HEIFU\textsuperscript{\textregistered} UAV. \\
		\midrule
		\multirow{2}*{\textbf{Remarks:}} 	& \emph{A microcontroller integrated circuit from the STM32} \\
							  				& \emph{family should be the base for the MCU.} \\
		\bottomrule
		&\\
		&\\
        \toprule
		\textbf{RTKBS.MAIN.FCT.040} 		& \textbf{Wireless Link} \\
		\multirow{4}*{}						& The RTK Base Station shall be able to wirelessly \\
											& transmit the differential correction information to the \\
											& devices around. \\
		\midrule
		\multirow{1}*{\textbf{Remarks:}} 	& \\
		\bottomrule
		&\\
		&\\
        \toprule
		\textbf{RTKBS.MAIN.FCT.050} 		& \textbf{Wireless Link Range} \\
		\multirow{1}*{}						& The wireless link shall reach at least 1 km VLOS. \\
		\midrule
		\multirow{1}*{\textbf{Remarks:}} 	& \\
		\bottomrule
	\end{tabular}
\end{table}
\endgroup

\subsection{Mechanical Requirements}\label{III:MEC_requirements}

This section describes the main mechanical requirements of the device.

\begingroup
\begin{table}[H]
	\captionsetup{justification=centering}
    \caption{beRTK\textsuperscript{\textregistered} Base Station mechanical requirements.}
	\label{tab:MEC_requirements}
	\centering

	% mini tabela: 12, 13, 14
	\begin{tabular}{rl}
        \toprule
	    \textbf{RTKBS.MAIN.MEC.010} 		& \textbf{Enclosure} \\
	    \multirow{5}*{}						& The RTK Base Station shall count on a plastic enclosure \\
											& as a shelter from the environment. This enclosure should \\
											& be designed in a way that is does not interfere with the \\
											& radio signals needed for its operation (namely \\
											& positioning and wireless communications).\\
		\midrule
		\multirow{1}*{\textbf{Remarks:}}   & \\
		\bottomrule
		&\\
		&\\
		\toprule
		\textbf{RTKBS.MAIN.MEC.020} 		& \textbf{Tripod Attachment} \\
		\multirow{3}*{}						& The RTK Base Station's enclosure shall be designed in \\
											& order to be possible to attach it to the top of a tripod \\
											& with a 5/8'' screw. \\
		\midrule
		\multirow{1}*{\textbf{Remarks:}} 	& \\
		\bottomrule
		&\\
		&\\
        \toprule
		\textbf{RTKBS.MAIN.MEC.030} 		& \textbf{Minimize Connectors} \\
		\multirow{3}*{}						& The RTK Base Station shall be designed so to minimize\\
											& the number of connectors to the outside world. This \\
											& includes antennae. \\
		\multirow{3}*{\textbf{Remarks:}}    & \emph{The idea is to ease and accelerate the deployment on the}\\
											& \emph{field, as well as minimize the vulnerability to}\\
											& \emph{external parameters.} \\
		\bottomrule
	\end{tabular}
\end{table}
\endgroup

\subsection{Electrical Requirements}\label{III:PWS_requirements}

This section describes the main electrical requirements of the device.

\begingroup
\begin{table}[H]
	\captionsetup{justification=centering}
    \caption{beRTK\textsuperscript{\textregistered} Base Station electrical requirements.}
	\label{tab:PWS_requirements}
	\centering

	% mini tabela: 15, 16, 17, 18, 19
	\begin{tabular}{rl}
        \toprule
		\textbf{RTKBS.MAIN.PWS.010} 			& \textbf{Main Power Supply} \\
		\multirow{6}*{}							& The RTK Base Station shall run on electricity, being \\
												& the main power source a connection to an external power \\
												& supply. It shall accept voltages between +7 VDC and \\
												& +22 VDC, enabling the connection to a standard solar \\
												& panel, a car battery or a mains power supply capable of \\
												& delivering a voltage within that range. \\
		\midrule
		\multirow{1}*{\textbf{Remarks:}}   & \\
		\bottomrule
		&\\
		&\\
		\toprule
		\textbf{RTKBS.MAIN.PWS.020} 		& \textbf{Battery} \\
		\multirow{4}*{}						& The RTK Base Station shall count on a single battery, \\
											& enabling it to be used on the field where a mains power \\
											& connection is not available. This battery shall be \\
											& embedded on the devices electronic system. \\
		\midrule
		\multirow{1}*{\textbf{Remarks:}} 	& \\
		\bottomrule
		&\\
		&\\
        \toprule
		\textbf{RTKBS.MAIN.PWS.030} 		& \textbf{Power Supply Switching} \\
		\multirow{5}*{}						& Having two separate power sources, the RTK Base Station \\
											& shall prioritize external power input, switching to the \\
											& internal battery when that connection cannot be \\
											& established. This transition must be smooth in order to \\
											& ensure continuous operation. \\
		\midrule
		\multirow{1}*{\textbf{Remarks:}} 	& \\
		\bottomrule
		&\\
		&\\
        \toprule
		\textbf{RTKBS.MAIN.PWS.040} 		& \textbf{Power Consumption} \\
		\multirow{3}*{}						& The RTK Base Station shall not exceed an average of \\
											& 400 mA of current consumption at 5 VDC voltage level. \\
											& That is, it shall not exceed a power consumption of 2 W.\\
		\midrule
		\multirow{1}*{\textbf{Remarks:}} 	& \\
		\bottomrule
		&\\
		&\\
        \toprule
		\textbf{RTKBS.MAIN.PWS.050} 		& \textbf{Power Switch} \\
		\multirow{3}*{}						& It shall be possible to turn the RTK Base Station on and\\
											& off by means of a pushbutton switch in toggle mode. This \\
											& button shall be externally available to the user. \\
		\midrule
		\multirow{1}*{\textbf{Remarks:}} 	& \\
		\bottomrule
	\end{tabular}
\end{table}
\endgroup

\subsection{Communication Requirements}\label{III:COM_requirements}

This section describes the main communication requirements of the device.

\begingroup
\begin{table}[H]
	\captionsetup{justification=centering}
    \caption{beRTK\textsuperscript{\textregistered} Base Station communication requirements.}
	\label{tab:COM_requirements}
	\centering

	% mini tabela: 20, 21, 22
	\begin{tabular}{rl}
        \toprule
		\textbf{RTKBS.MAIN.COM.010} 			& \textbf{Communication with NTRIP Platform} \\
		\multirow{3}*{}							& The RTK Base Station shall be capable of establishing \\
												& communications with an NTRIP platform by using a \\
												& Wi-Fi radio link. \\
		\midrule
		\multirow{1}*{\textbf{Remarks:}}   & \\
		\bottomrule
		&\\
		&\\
		\toprule
		\textbf{RTKBS.MAIN.COM.020} 		& \textbf{Communication with the UAVs} \\
		\multirow{3}*{}						& The RTK Base Station shall be capable of establishing \\
											& wireless communications with the UAVs by using a \\
											& well-established local communication radio protocol. \\
		\midrule
		\multirow{2}*{\textbf{Remarks:}} 	& \emph{Communication can be established either using the} \\
											& \emph{IEEE 802.15.4 or LoRa protocols.}\\
		\bottomrule
		&\\
		&\\
        \toprule
		\textbf{RTKBS.MAIN.COM.030} 		& \textbf{Communication with the User's Computer} \\
		\multirow{3}*{}						& The RTK Base Station shall be capable of establishing \\
											& communications with an external computer by using a \\
											& Wi-Fi radio link. \\
		\midrule
		\multirow{1}*{\textbf{Remarks:}} 	& \\
		\bottomrule
	\end{tabular}
\end{table}
\endgroup

\subsection{User Interface Requirements}\label{III:HMI_requirements}

This section describes the main user interface requirements of the device.

\begingroup
\begin{table}[H]
	\captionsetup{justification=centering}
    \caption{beRTK\textsuperscript{\textregistered} Base Station user interface requirements.}
	\label{tab:HMI_requirements}
	\centering

	% mini tabela: 23, 24, 25, 26
	\begin{tabular}{rl}
        \toprule
		\textbf{RTKBS.MAIN.HMI.010} 			& \textbf{Power Button} \\
		\multirow{2}*{}							& The RTK Base Station shall exhibit, on its front panel, \\
												& the power button. \\
		\midrule
		\multirow{1}*{\textbf{Remarks:}}   & \\
		\bottomrule
		&\\
		&\\
		\toprule
		\textbf{RTKBS.MAIN.HMI.020} 		& \textbf{Battery Status} \\
		\multirow{3}*{}						& The RTK Base Station shall exhibit, on its front panel, \\
											& the status of the power source currently being used, by \\
											& means of four blue LEDs. \\
		\midrule
		\multirow{1}*{\textbf{Remarks:}} 	& \\
		\bottomrule
		&\\
		&\\
        \toprule
		\textbf{RTKBS.MAIN.HMI.030} 		& \textbf{Power Source Indication} \\
		\multirow{3}*{}						& The RTK Base Station shall exhibit, on its front panel, \\
											& an indication of which power sources are available, by \\
											& means of three blue LEDs. \\
		\midrule
		\multirow{1}*{\textbf{Remarks:}} 	& \\
		\bottomrule
		&\\
		&\\
        \toprule
		\textbf{RTKBS.MAIN.HMI.040} 		& \textbf{Power Connector} \\
		\multirow{3}*{}						& The RTK Base Station shall exhibit, on its back panel, \\
											& one connector enabling the connection to an external \\
											& power source. \\
		\midrule
		\multirow{1}*{\textbf{Remarks:}} 	& \emph{On this stage, the connector should be USB Type-C.} \\
		\bottomrule
	\end{tabular}
\end{table}
\endgroup

\subsection{Software Requirements}\label{III:SW_requirements}

This section describes the main software requirements of the device.

\begingroup
\begin{table}[H]
	\captionsetup{justification=centering}
    \caption{beRTK\textsuperscript{\textregistered} Base Station software requirements.}
	\label{tab:SW_requirements}
	\centering

	% mini tabela: 27, 28, 29, 30
	\begin{tabular}{rl}
        \toprule
		\textbf{RTKBS.MAIN.SW.010} 				& \textbf{Operating system} \\
		\multirow{2}*{}							& The RTK Base Station MCU shall be able to run an \\
												& embedded operating system. \\
		\midrule
		\multirow{1}*{\textbf{Remarks:}}   & \\
		\bottomrule
		&\\
		&\\
		\toprule
		\textbf{RTKBS.MAIN.SW.020} 			& \textbf{RTK Local Web Platform} \\
		\multirow{2}*{}						& The RTK Base Station MCU shall have a Local Web \\
											& Platform that controls the RTK function. \\
		\midrule
		\multirow{1}*{\textbf{Remarks:}} 	& \\
		\bottomrule
		&\\
		&\\
        \toprule
		\textbf{RTKBS.MAIN.SW.030} 			& \textbf{Configuration} \\
		\multirow{4}*{}						& The RTK base Station shall include a configuration \\
											& interface, enabling a user to configure the base \\
											& parameters. This interface shall be available via the \\
											& RTK Local Web Platform. \\
		\midrule
		\multirow{1}*{\textbf{Remarks:}} 	& \\
		\bottomrule
		&\\
		&\\
        \toprule
		\textbf{RTKBS.MAIN.SW.040} 			& \textbf{Logs} \\
		\multirow{3}*{}						& The RTK base Station shall record the main activities \\
											& in a log file, so it can be possible to back trace its \\
											& behaviour. \\
		\midrule
		\multirow{1}*{\textbf{Remarks:}} 	& \\
		\bottomrule
	\end{tabular}
\end{table}
\endgroup
