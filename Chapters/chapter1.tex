%!TEX root = ../template.tex
%%%%%%%%%%%%%%%%%%%%%%%%%%%%%%%%%%%%%%%%%%%%%%%%%%%%%%%%%%%%%%%%%%%
%% chapter1.tex
%% NOVA thesis document file
%%
%% Chapter with introduction
%%%%%%%%%%%%%%%%%%%%%%%%%%%%%%%%%%%%%%%%%%%%%%%%%%%%%%%%%%%%%%%%%%%

% \prependtographicspath{{Chapters/Figures/Covers/phd/}{Chapters/Figures/Covers/msc/}}

% epigraph configuration
% \epigraphfontsize{\small\itshape}
% \setlength\epigraphwidth{12.5cm}
% \setlength\epigraphrule{0pt}

% \epigraph{
%   This work is licensed under the \href{LaTeX project public license}{\LaTeX\ Project Public License v1.3c}.
%   To view a copy of this license, visit \url{LaTeX project public license}.
% }

\typeout{NT FILE chapter1.tex}

\chapter{Description of the Problem}\label{cha:introduction_description}

Around twelve thousand years ago, the way of life of the average human being changed completely (until then, it simply followed primitive and sometimes unproductive ways of hunting). The domestication of plants and breeding of animals started the Neolithic period, which paved the way for the revolution and subsequent development of agriculture. %_citation needed?

Over the years, agricultural exploration has made the discovery and invention of countless phenomena and techniques possible, many of which are still practised and improved today, since this is, precisely, an archaic field by its very nature.
However, the doors that lead to this type of exploration are far from being closed; new devices, able to efficiently complement and facilitate human work in the agricultural sector, are still created and improved to this very day. Thus, when developing a machine capable of performing tasks that, normally, a human would do, one of the starting points to focus on are often the simulation of human senses. Sight is (arguably), the most important one to consider in 

But there is still a great ally for such efficiency, which sometimes goes unnoticed or is even forgotten: Image Processing.

\section{Section 1.1}\label{sub:sub1_1}

\section{Section 1.2}\label{sec:sub1_2}

\section{Section 1.3}\label{sec:sub1_3}

\section{Section 1.4}\label{sec:sub1_4}