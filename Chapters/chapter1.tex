%!TEX root = ../template.tex
%%%%%%%%%%%%%%%%%%%%%%%%%%%%%%%%%%%%%%%%%%%%%%%%%%%%%%%%%%%%%%%%%%%
%% chapter1.tex
%% NOVA thesis document file
%%
%% Chapter with introduction
%%%%%%%%%%%%%%%%%%%%%%%%%%%%%%%%%%%%%%%%%%%%%%%%%%%%%%%%%%%%%%%%%%%

% \prependtographicspath{{Chapters/Figures/Covers/phd/}{Chapters/Figures/Covers/msc/}}

% epigraph configuration
% \epigraphfontsize{\small\itshape}
% \setlength\epigraphwidth{12.5cm}
% \setlength\epigraphrule{0pt}

% \epigraph{
%   This work is licensed under the \href{LaTeX project public license}{\LaTeX\ Project Public License v1.3c}.
%   To view a copy of this license, visit \url{LaTeX project public license}.
% }

\typeout{NT FILE chapter1.tex}

\chapter{Introduction}\label{cha:I_introduction}

%mudar o começo desta frase -> tirar os parenteses
Around twelve thousand years ago, the way of life of the average human being -- which, until then, simply followed primitive and sometimes unproductive ways of hunting -- changed completely. The domestication of plants and breeding of animals started in the Neolithic period, which paved the way for the revolution and subsequent development of agriculture~\cite{Zeder2011TheOO}.
%juntar este paragrafos com o do "Over the years,"

The invention of the cotton gin in 1792 introduced the first ever mechanisation of an agricultural process, resulting immediately in enormous volumes of production, quickly enriching societies around the globe~\cite{Roe1926}.
This resulted in a new age in agriculture that continues to this day, where machines are created and improved in order to efficiently complement and facilitate human work in almost every process possible.
However, some tasks remain relatively tedious (such as pesticide sprayers mounted on tractors) or are just too inefficient, despite the use of machines. Robotics-oriented devices are able to patch the holes in both cases, notably in the latter.
%falar de infraestruturas e segurança agora
It is also quite obvious that work of this nature is not exclusive to the agricultural sector; there are thousands of people responsible for jobs regarding an important and sometimes overlooked need of society: surveillance.
Taking the example of developing projects % ir ao site da aeriya ver o que o drone faz; meter aqui
workplace accidents or, in a more drastic approach, Siberia's sadly expected annual wildfires (which, in 2021, reported the most devastating one of the 21$^{st}$ century~\cite{dixon_2021,roth_2021}), the mitigation of situations that could, possibly, negatively impact human life is undoubtedly imperative.%%%%%%%%%%%%%%%%%

With rapid technological advancement, a factor foreseen to be virtually removed was, precisely, the human contribution. Thus, when developing a device capable of performing tasks that, normally, a human would be responsible for, the main focus should be envisioning how each of those could be performed (approached?) by it.

Sight is more than often the most important sense to consider, therefore, bearing all of this in mind, a device capable of ``seeing'', acknowledging and mapping an area of work would be an excellent starting point for the previously mentioned sectors. Nowadays, tasks like these can be performed by devices known as ``Unmanned Aerial Vehicles'' (or UAVs), designed for a vast array of functions.

% OU meter este paragrafo la em baixo, no fim da introduçao.
% SERÁ QUE ESTE PARAGRAFO ESTA A MAIS?
% Over the years, the discovery of countless phenomena as well as the development of numerous techniques have been made possible by exploring this laborous sector, and the doors that lead to the exploration seem far from being closed.

\section{Description of the Problem}\label{sec:I_description}
% falar dos objetivos da beyond
% Agriculture:
%              Multispectral Data Collection
%              Creation of Aerial Maps
%              Data Evaluation
% Infrastructure:
%              Data Collection
%              Production of Aerial Maps
%              Project Management
% Safety:
%              Autonomous Patrol
%              Danger Zone Patrol
%              Occurrence Control Management
%%%% 1. qual o problema a solucionar?
%%%% 1.1. abordar aplicações para drones ou outros problemas a resolver (eg aquela cena do trator com a base station por cima, ou um veiculo autonomo... a base pode servir para mais coisas sem ser so para drones)

%%%% 1.2. ilustrar o prblema - eg fotos de terrenos grandes e necessidade de precisao (nesse caso agricultura de precisao), ou fotos de carros autonomos e necessidade de precisao (nesse caso segurança)

%%%% 2. analise da concorrencia no final da abordagem do problema - vao se selecionar 6 colunas (specs) da tabela do excel que eu fiz (o pedro vai selecionar) e eu tenho de escolher 4 soluções (base stations) (5 no máximo) que possa usar como guidelines para a realização do meu trabalho

When using UAVs, constantly having to manually control them in tedious and lengthy tasks (e.g. agricultural field surveying) might be impractical.
A solution for this type of problem would be to automate the processes assigned to the devices. For that, one can't only rely on GPS for extremely accurate positioning, specially in a real time scenario, since this service does not provide the level of accuracy needed for precise navigation, which must be in the order of centimeters. Real-time kinematics (RTK) solves this issue~\cite{gps_USGov,8714161},
since it is a technology designed to improve the accuracy of GNSS receivers in real time. This means it perfecty fits the high percision need of autonomous vehicles, imperative to carry out certain missions with virtually to no human assistance. Therefore, high accuracy is absolutely necessary in real-time UAV positioning.
% AND LANDING OF THE DRONE ?????????????
However, as this type of equipment is in constant motion, the problem of obtaining highly accurate position readings in real time arises, which in turn calls for a solution that is capable of providing such readings -- a base station.

\subsection{What is a Base Station?}\label{sec:I_base_station}

A UAV used in precision agriculture or surveillance, for example, usually has a GNSS receiver integrated in its board which, just like a regular smartphone, helps tracking down its current position. However, as mentioned in section~\ref{sec:I_description}, the offered
precision is not the best for real-time applications. To solve this issue, a base station can be used as a position adjustment auxiliary.
Also called a GNSS receiver itself, a base station is a stationary equipment that, through a link to the vast array of satellites arranged in outer space (also known as constellations), is able to send position corrections to another GNSS receiver linked to it (usually a rover -- a UAV in this case).
In order for this to be possible, the base station must have information regarding its exact position, which can be determined through either a GNSS constelation or NTRIP network (explained in more detail in chapter~\ref{cha:II_SotA}) % DÚVIDA: só há estas 2 maneiras?
it is required that the base station having a previously fixed and known position (transcrever video do tropa) (i.e. never moving). This enables it to confirm the exact position of the rover in an extremely accurate, real-time manner.

in this case, the rtk base on which to work (beRTK) has 2 major functions: the first is to use the u-blox ZED-F9P to obtain a highly accurate fixed position and transmit RTK corrections via ZigBee for the Drone (HEIFU);
the second is to connect to a ReNEP NTRIP server, in order to obtain a precise location much faster (by pure GPS, as said, it takes a long time, it can up to hours, until an accuracy below 10cm (REFERENCE NEEDED) is obtained, with the help of NTRIP it takes just minutes).
The idea proposed in this dissertation is to find a lower power solution than the one that Beyond Vision currently owns. What currently exists makes use of a Raspberry Pi Model 4B (micro)computer with two WiFi modules:
one to connect to a WiFi network and effectively go to the NTRIP fetch data server;
other to create a local network that has, in turn, a Web server with a base configuration page (RTCM and scenes like that, SEE beRTK DATASHEET). It is therefore intended to use one (or two) microcontrollers of in order to perform these functions, keeping the ZED-F9P and ZigBee.
They are fine on the change list, but with lower priority, so a more robust battery system, something that allows charging and using the base at the same time (preferably something that charges via USB-C).
% enter a NUTSHELL setting of RTK,
a base station is intended that is capable of fine-tuning the position of the rover (drone);

% |=|=|=|=|=|=|=|=|=|
But there is still a great ally for such efficiency, which sometimes goes unnoticed is even forgotten: Image Processing.

Começar com:
I. o que é uma base station? fazer analogia
II. dizer para que e que serve
III. onde/no que é que eu vou empregá-la?
IV. que tecnologias utiliza?
V. explicar cada tecnologia relativamente a base

Talk about:
    1. Satellite Navigation Device
    2. Transceiver
    3. Base station
    4. Aerial base station
    5. how GPS works - https://electronics.howstuffworks.com/gadgets/travel/gps.htm
    6. how satellites work - https://science.howstuffworks.com/satellite.htm
    7. atomic clocks - https://science.howstuffworks.com/atomic-clock.htm

    8. differential GPS (DGPS) - The term differential GPS, or DGPS, sometimes indicates the application of this technique with coded pseudorange measurements
    8.1. relative GPS - usually indicates the application of this technique with carrier phase measurements
    8.2. carrier phase measurements
    8.3. baselines
    in: https://www.e-education.psu.edu/geog862/node/1725

    9. GNSS - The performance of GNSS is assessed using four criteria: Accuracy, Integrity, Continuity and Availability. The correlated range errors due to ephemeris prediction errors and residual satellite clock, ionosphere and troposphere errors may vary slowly with time and user location.
    Therefore, by comparing pseudo-range measurements with those made by equipment at a presurveyed location, known as a REFERENCE STATION or BASE STATION, the correlated range errors may be calibrated out, improving the navigation-solution Accuracy! This is the priciple behind Differential GNSS (DGNSS)~\cite{edseee_9101092}. % fazer desenho .svg de base station a "acertar" o drone com os satelites
    10. L2C and L5 (signal availability) bands;
    11. the future L1C
    12. Capacity of continuous measurements
    13. static precision measurement
    14. dynamic precision measurement
    15. what is a 120-channel receiver?
    16. explicar o que cada célula do excel significa, tanto as que considerei mais importantes como as outras
    17. RTK / RTK-GNSS / D-RTK / dynamic differential technology~\cite{ayers_geosystems_2011}
       
    land surveying - In the context of external land surveying, a base station is a GPS receiver at an accurately-known fixed location which is used to derive correction information for nearby portable GPS receivers. This correction data allows propagation and other effects to be corrected out of the position data obtained by the mobile stations, which gives greatly increased location precision and accuracy over the results obtained by uncorrected GPS receivers.

    how long does it take for a radio signal to be emitted from a satellite to reach the surface of the earth? 
    R.: given that a satellite circles the globe at an altitude of about 19.3 km --> d = v*t => 19.3*10$^3$ / 300 000 000 = t = 6.43*10$^{-5}$ = 64.3 us.

% artigos que li:
a. Experimental Testbed and Methodology for the
Assessment of RTK GNSS Receivers Used
in Precision Agriculture;

b. DETERMINATION OF THE POSITION USING
RECEIVERS INSTALLED IN UAV

c. High-Precision/Throughput Growth Measurement of
Crops by Drone with Stereo Matching Based on
RTK-GNSS and Single Camera

d. Estimation of the Base Station Position Error in a
RTK Receiver Using State Augmentation in a
Kalman Filter

e. Resilient Deployment of Drone Base Stations

f. Based on a single-base station RTK control survey
and precision analysis 

g. Design of an Autonomous drone for IoT deployment
analysis 

h. RTK+ System for Precise Navigation in Shadowed
Areas 
