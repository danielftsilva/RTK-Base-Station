%!TEX root = ../template.tex
%%%%%%%%%%%%%%%%%%%%%%%%%%%%%%%%%%%%%%%%%%%%%%%%%%%%%%%%%%%%%%%%%%%
%% chapter1.tex
%% NOVA thesis document file
%%
%% Chapter with introduction
%%%%%%%%%%%%%%%%%%%%%%%%%%%%%%%%%%%%%%%%%%%%%%%%%%%%%%%%%%%%%%%%%%%

% \prependtographicspath{{Chapters/Figures/Covers/phd/}{Chapters/Figures/Covers/msc/}}

% epigraph configuration
% \epigraphfontsize{\small\itshape}
% \setlength\epigraphwidth{12.5cm}
% \setlength\epigraphrule{0pt}

% \epigraph{
%   This work is licensed under the \href{LaTeX project public license}{\LaTeX\ Project Public License v1.3c}.
%   To view a copy of this license, visit \url{LaTeX project public license}.
% }

\typeout{NT FILE chapter1.tex}

\chapter{Description of the Problem}\label{cha:introduction_description}

Around twelve thousand years ago, the way of life of the average human being changed completely (until then, it simply followed primitive and sometimes unproductive ways of hunting). The domestication of plants and breeding of animals started the Neolithic period, which paved the way for the revolution and subsequent development of agriculture. %_citation needed?

Over the years, agricultural exploration has made the discovery and invention of countless phenomena and techniques possible, many of which are still practised and improved today, since this is, precisely, an archaic field by its very nature.
However, the doors that lead to this type of exploration are far from being closed; new devices, able to efficiently complement and facilitate human work in the agricultural sector, are still created and improved to this very day. Thus, when developing a machine capable of performing tasks that, normally, a human would do, one of the starting points to focus on are often the simulation of human senses. Sight is (arguably), the most important one to consider in 

But there is still a great ally for such efficiency, which sometimes goes unnoticed or is even forgotten: Image Processing.



Talk about:
    1. Satellite Navigation Device
    2. Transceiver
    3. Base station
    4. Aerial base station
    5. how GPS works - https://electronics.howstuffworks.com/gadgets/travel/gps.htm
    6. how satellites work - https://science.howstuffworks.com/satellite.htm
    7. atomic clocks - https://science.howstuffworks.com/atomic-clock.htm

    8. differential GPS (DGPS) - The term differential GPS, or DGPS, sometimes indicates the application of this technique with coded pseudorange measurements
    8.1. relative GPS - usually indicates the application of this technique with carrier phase measurements
    8.2. carrier phase measurements
    8.3. baselines
    in: https://www.e-education.psu.edu/geog862/node/1725

    9. GNSS
    10. L2C and L5 (signal availability) bands;
    11. the future L1C
    12. Capacity of continuous measurements
    13. static precision measurement
    14. dynamic precision measurement
    15. what is a 120-channel receiver?
    16. explicar o que cada célula do excel significa, tanto as que considerei mais importantes como as outras
    17. RTK / RTK-GNSS / D-RTK / dynamic differential technology
        

    land surveying - In the context of external land surveying, a base station is a GPS receiver at an accurately-known fixed location which is used to derive correction information for nearby portable GPS receivers. This correction data allows propagation and other effects to be corrected out of the position data obtained by the mobile stations, which gives greatly increased location precision and accuracy over the results obtained by uncorrected GPS receivers.

    how long does it take for a radio signal to be emitted from a satellite to reach the surface of the earth? 
    R.: given that a satellite circles the globe at an altitude of about 19.3 km --> d = v*t => 19.3*10$^3$ / 300 000 000 = t = 6.43*10$^{-5}$ = 64.3 us.

% artigos que li:
a. Experimental Testbed and Methodology for the
Assessment of RTK GNSS Receivers Used
in Precision Agriculture;

b. DETERMINATION OF THE POSITION USING
RECEIVERS INSTALLED IN UAV

c. High-Precision/Throughput Growth Measurement of
Crops by Drone with Stereo Matching Based on
RTK-GNSS and Single Camera

d. Estimation of the Base Station Position Error in a
RTK Receiver Using State Augmentation in a
Kalman Filter

e. Resilient Deployment of Drone Base Stations

f. Based on a single-base station RTK control survey
and precision analysis 

g. Design of an Autonomous drone for IoT deployment
analysis 

h. RTK+ System for Precise Navigation in Shadowed
Areas 

\section{Section 1.1}\label{sub:sub1_1}

\section{Section 1.2}\label{sec:sub1_2}

\section{Section 1.3}\label{sec:sub1_3}

\section{Section 1.4}\label{sec:sub1_4}