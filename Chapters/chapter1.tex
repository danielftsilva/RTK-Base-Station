%!TEX root = ../template.tex
%%%%%%%%%%%%%%%%%%%%%%%%%%%%%%%%%%%%%%%%%%%%%%%%%%%%%%%%%%%%%%%%%%%
%% chapter1.tex
%% NOVA thesis document file
%%
%% Chapter with introduction
%%%%%%%%%%%%%%%%%%%%%%%%%%%%%%%%%%%%%%%%%%%%%%%%%%%%%%%%%%%%%%%%%%%

% \prependtographicspath{{Chapters/Figures/Covers/phd/}{Chapters/Figures/Covers/msc/}}

% epigraph configuration
% \epigraphfontsize{\small\itshape}
% \setlength\epigraphwidth{12.5cm}
% \setlength\epigraphrule{0pt}

% \epigraph{
%   This work is licensed under the \href{LaTeX project public license}{\LaTeX\ Project Public License v1.3c}.
%   To view a copy of this license, visit \url{LaTeX project public license}.
% }

\typeout{NT FILE chapter1.tex}

\chapter{Introduction}\label{cha:I_introduction}

%mudar o começo desta frase -> tirar os parenteses
Around twelve thousand years ago, the way of life of the average human being -- which, until then, simply followed primitive and sometimes unproductive ways of hunting -- changed completely. The domestication of plants and breeding of animals started in the Neolithic period, which paved the way for the revolution and subsequent development of agriculture~\cite{Zeder2011TheOO}.
%juntar este paragrafos com o do "Over the years,"

The invention of the cotton gin in 1792 introduced the first ever mechanisation of an agricultural process, resulting immediately in enormous volumes of production, quickly enriching societies around the globe~\cite{Roe1926}.
This resulted in a new age in agriculture that continues to this day, where machines are created and improved in order to efficiently complement and facilitate human work in almost every process possible.
However, some tasks remain relatively tedious (such as pesticide sprayers mounted on tractors) or are just too inefficient, despite the use of machines. Robotics-oriented devices are able to patch the holes in both cases, notably in the latter.
%falar de infraestruturas e segurança agora
It is also quite obvious that work of this nature is not exclusive to the agricultural sector; there are thousands of people responsible for jobs regarding an important and sometimes overlooked need of society: surveillance.
Taking the example of developing projects % ir ao site da aeriya ver o que o drone faz; meter aqui
workplace accidents or, in a more drastic approach, Siberia's sadly expected annual wildfires (which, in 2021, reported the most devastating one of the 21$^{st}$ century~\cite{dixon_2021,roth_2021}), the mitigation of situations that could, possibly, negatively impact human life is undoubtedly imperative.%%%%%%%%%%%%%%%%%

With rapid technological advancement, a factor foreseen to be virtually removed was, precisely, the human contribution. Thus, when developing a device capable of performing tasks that, normally, a human would be responsible for, the main focus should be envisioning how each of those could be performed (approached?) by it.

Sight is more than often the most important sense to consider, therefore, bearing all of this in mind, a device capable of ``seeing'', acknowledging and mapping an area of work would be an excellent starting point for the previously mentioned sectors. Nowadays, tasks like these can be performed by devices known as ``Unmanned Aerial Vehicles'' (or UAVs), designed for a vast array of functions.

% OU meter este paragrafo la em baixo, no fim da introduçao.
% SERÁ QUE ESTE PARAGRAFO ESTA A MAIS?
% Over the years, the discovery of countless phenomena as well as the development of numerous techniques have been made possible by exploring this laborous sector, and the doors that lead to the exploration seem far from being closed.

\section{Description of the Problem}\label{sec:I_description}
% falar dos objetivos da beyond
% Agriculture:
%              Multispectral Data Collection
%              Creation of Aerial Maps
%              Data Evaluation
% Infrastructure:
%              Data Collection
%              Production of Aerial Maps
%              Project Management
% Safety:
%              Autonomous Patrol
%              Danger Zone Patrol
%              Occurrence Control Management
%%%% 1. qual o problema a solucionar?
%%%% 1.1. abordar aplicações para drones ou outros problemas a resolver (eg aquela cena do trator com a base station por cima, ou um veiculo autonomo... a base pode servir para mais coisas sem ser so para drones)

%%%% 1.2. ilustrar o prblema - eg fotos de terrenos grandes e necessidade de precisao (nesse caso agricultura de precisao), ou fotos de carros autonomos e necessidade de precisao (nesse caso segurança)

%%%% 2. analise da concorrencia no final da abordagem do problema - vao se selecionar 6 colunas (specs) da tabela do excel que eu fiz (o pedro vai selecionar) e eu tenho de escolher 4 soluções (base stations) (5 no máximo) que possa usar como guidelines para a realização do meu trabalho

When using UAVs, constantly having to manually control them in tedious and lengthy tasks (e.g. agricultural field surveying) might be impractical.
A solution for this type of problem would be to automate the processes assigned to the devices. For that, one can't only rely on GPS for extremely precise positioning, specially in a real time scenario, since this service does not provide the level of accuracy needed for precise navigation, which must be in the order of centimeters. Real-time kinematics (RTK) solves this issue~\cite{gps_USGov,8714161},
since it is a technology designed to improve the accuracy of GNSS receivers in real time. This means it perfecty fits the high percision need of autonomous vehicles, imperative to carry out certain missions with virtually to no human assistance. Therefore, high accuracy is absolutely necessary in real-time UAV positioning.
% AND LANDING OF THE DRONE ?????????????
However, as this type of equipment is in constant motion, the problem of obtaining highly precise position readings in real time arises, which in turn calls for a solution that is capable of providing such readings -- a base station.

%\subsection{What is a Base Station?}\label{sec:I_base_station}
A UAV used in precision agriculture or surveillance, for example, usually includes a GNSS receiver integrated in its board which, just like a regular smartphone, helps tracking down its current position. However, as mentioned in section~\ref{sec:I_description}, the offered precision is not the best for real-time applications.
To solve this issue, an equipment known as ``base station'' can be used as a position adjustment auxiliary.
Also containing a GNSS receiver itself, a base station is stationary and, through a link to the vast array of satellites arranged in outer space (also known as constellations), is able to send position corrections to another GNSS receiver linked to it (usually a rover, e.g. a UAV), in a real-time manner.
In order for this to be possible, the base station must be fixed and have information regarding its exact position, which can be determined directly from a GNSS constellation or through the internet.
As soon as that is established, the corrections made by it will be extremely trustworthy and precise.
%(explained in more detail in chapter~\ref{cha:II_SotA})
% DÚVIDA: só há estas 2 maneiras?

% The Problem to Solve
This dissertation aims at the redesign of the RTK base station produced by the company in collaboration with\footnote{Beyond Vision.}. This product is known as beRTK\textsuperscript{\textregistered}~\cite{beRTK_2022} and, currently, has two major \textbf{(main?)} functions: the first is to use a high precision GNSS module named \textbf{ZED-F9P}, in order to obtain its position in a highly precise manner and then broadcasting RTK corrections via \textbf{ZigBee technology} to a UAV; % falar do HEIFU?
the second is to connect itself to an NTRIP network in order to obtain its precise location much faster (as, through pure GPS positioning, obtaining a precision value below 10 centimeters takes a long time -- up to hours; relying on NTRIP reduces that same waiting time to just minutes or even seconds).
The proposed idea \textbf{(in this dissertation?)} is to work on finding a solution that consumes \textbf{(uses?)} less power than the one currently developed.
%low-power solution than the one that Beyond Vision currently owns.

The current solution manages WiFi connections through a \textbf{Raspberry Pi 4 Model B}, a \textbf{single-board computer (SBC)} with two WiFi modules:
one able to connect the base station to a WiFi network and fetch positioning data from the NTRIP network, and another that enables the remote configuration of the base from a web-based application, through the creation of a local network.
It is, therefore, intended to replace the \textbf{SBC} used with microcontrollers able to perform these same operations, while keeping both \textbf{ZED-F9P} and \textbf{ZigBee} modules.
Also on the list of changes, yet with lower priority, is a more robust battery system -- one \textbf{(an option?)} that allows charging and using the base at the same time (preferably via \textbf{USB-C}). The power consumption of this sector is intended to be cut in approximately half.
