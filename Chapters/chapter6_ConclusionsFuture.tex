%!TEX root = ../template.tex
%%%%%%%%%%%%%%%%%%%%%%%%%%%%%%%%%%%%%%%%%%%%%%%%%%%%%%%%%%%%%%%%%%%%
%% chapter6_ConclusionsFuture.tex
%% NOVA thesis document file
%%
%% Chapter with the Conclusions and Future Work part
%%%%%%%%%%%%%%%%%%%%%%%%%%%%%%%%%%%%%%%%%%%%%%%%%%%%%%%%%%%%%%%%%%%%

\typeout{NT FILE chapter6_ConclusionsFuture.tex}

\chapter{Conclusions and Future Work}\label{cha:chapter6_ConclusionsFuture}

% Considering the impact of robotic devices in the most diverse fields, their precise development is of great importance, so the definition and implementation of a project in an orderly and careful way is quite valuable.

Supported by a strong definition of the basic concepts behind the operation of a base station, the development of new beRTK\textsuperscript{\textregistered} proved to be interesting, exciting, and above all, challenging throughout its course. Analysing the state-of-the-art of base station design provided a better understanding of a device that may be viewed as a mere positioning aid tool. Designing each circuit that defined the base's system required the rediscovery of many topics, and opened up several gates to new ones in electronic product design. The specifications phase -- i.e. the definition of requirements to fulfil presented in Section\ref{sec:II_Specs} -- backed by the knowledge of a product's typical life cycle stages, helped divide the project into different phases, which were faithfully followed.

Regarding the hardware design phase, subdividing the system's architecture as presented in Figure~\ref{fig:architecture_new} proved as a valuable way to approach the design process. The first section of the project to be designed was the power supply chain. As the literal ``first'', critical member of the system, it is in charge of distributing power to its entirety, and thus it is imperative for this section to be as thought-out and efficient as possible. The design has to account for signal problems such as over or under-voltages and disturbances (e.g. noise) that may arise and propagate to the upstream circuits. It was also clear from the start that a low consumption of power by the device was required, and therefore not only the power supply section was designed with that in mind, but also the remaining system members. In other words, the base station's design focuses on the selection of components capable of fulfilling the necessary requirements while, at the same time, providing the minimum consumption of power possible.

With the power supply chain laid out, designing the remaining circuits for the control, peripherals, and communications revolved around their function-defining ICs selected in the beginning, while also looking for the best ways to tie all circuits together into a single, reliable system.

Upon completion of the schematic design, entering the realm of layout design provided new, exciting challenges, as other concerns had to be accounted for while defining a virtual circuit board that would soon become palpable. For that, different points of view were discovered for all the decisions that were undertaken in the schematic design phase. Finally, the prototype assembly solidified all the similarities and differences between both hardware design phases, resulting in a physical version of the system, ready for testing.

Turning the prototype on for the first time would settle all the questions and speculations that only the real world could answer. To obtain the necessary voltages for the correct functioning of the prototype, some resistor values were altered, which proved not to be a hassle, since the only work to redo fell solely on recalculation -- i.e. reusing the defined expressions.

However, design flaws resulted in problems that affected the device's performance -- which was concluded in the functional testing phase --, and must therefore be resolved upon working on future versions of the beRTK\textsuperscript{\textregistered}. By order of importance, the first rework to be made is regarding the BQ29209 battery balancer. 

%aquele circuito novo do bq29209:---------------------------------
%bq29209:
As mentioned in Section~\ref{sec:3212_BQ29209},~\cite{bq29209} provides a typical application circuit for the BQ29209 together with a few application notes. However, that typical application circuit should not be implemented for the new beRTK\textsuperscript{\textregistered}, since the internal system of the BQ29209 can only support cell balancing currents of up to 15mA, which would not cause any kind of reaction within the cells used (INR18650-35E), since their minimum charge current is 68mA. Therefore,~\cite{bq29209} provides another application circuit capable of supporting a cell balancing current higher than 15mA. Figure~\ref{fig:BQ29209_circuit_NEW} in Annex~\ref{ann:NEW_BQ29209_circuit} presents an adaptation for such implementation.

This new circuit is also valid for a $V_{DD}$ supply voltage ranging from 4V to 10V. It also ensures that, due to FETs Q3 (NMOS, 2N7002 model) and Q10 (PMOS, FDD6685 model), a higher cell balancing current can safely be established, since it is defined by either cell's voltage, along with resistor R4, i.e. $I_{BAL}=\frac{V_{Cell}}{R4}$.

% \begin{equation}\label{eq:IBAL}
% 	I_{BAL}=\frac{V_{Cell}}{R4}\,\medskip
% \end{equation}

\noindent It should be noted that the purpose for resistor R54 lies on ensuring that both ``balancing'' FETs will remain off while cell balancing is not taking place, and should be selected with a value above 2k$\Omega$.

\cite{bq29209} clarifies that the BQ29209 ensures cell balancing on the cell that presents the higher voltage of the pack.
Taking into account that the recommended cell arrangement sequence -- from the bottom of the stack -- of GND--Cell1--Cell2 (i.e. from pin 3 to pin 1 of connector F9 -- Figure~\ref{fig:BQ29209_circuit}) was followed, the balancing of Cell1 would start by pulling pin VC1\_CB of BQ29209 to GND, therefore setting PMOS Q10's gate-source voltage at around $-V_{Cell1}$. This would cause a balancing current of $I_{BAL1}=\frac{V_{Cell1}}{R4}$. Conversely, if Cell2 were in need of balancing, VC1\_CB would instead be pulled to $V_{DD}$, thus setting Q3's gate voltage at around $V_{Cell1}+V_{Cell2}$, which translates into a gate-source voltage of approximately $V_{Cell2}$, having the source voltage being set at around $V_{Cell1}$. With this, a cell balancing current of $I_{BAL2}=\frac{V_{Cell2}}{R4}$ results. In order to set a balancing current for each cell that would always surpass the minimum charge current of 68mA, a value of 20$\Omega$ would suffice. 

The remaining components' values can be selected as to comply with the BQ29209 recommended operating conditions.
%------- fim do new circuit do BQ29209 ----------------------

As mentioned in Section~\ref{sec:5331_CM4_UNmounted}, the VBAT net will never measure a voltage value anywhere close to zero, due to the voltage programming done for this LTC4012 pin. That proves to be an issue when operational Mode 1 is used, since the status LEDs remain lit up when this mode is in operation. The issue is also related to the clash between the voltage on the !ACP pin and the 5V net, as the 5V net will always be set, thus always setting FET Q8 into its active zone as well, even when the !ACP pin drops its voltage. This conflict could be resolved by using terminal 2 of the F9 battery pack connector (BAT2 net) and using a 2N7002 N-channel FET for Q8 instead of the NX3008NBK model used -- due to the 2N7002 ON characteristics~\cite{2N7002}, namely, its gate threshold voltage --, as well as a new N-channel FET (also a 2N7002 model) between terminal 1 of the same F9 connector and the drain of FET Q6. Figure~\ref{fig:BQ29209_circuit_NEW} also showcases this proposed solution.

The USB-related circuits must also be worked on in the future in order to solve the observed issues. The focus should be to effectively enable USB data transfer and carry on with the project's testing and eventual certifications. For that, as stated in the closing paragraph of Chapter~\ref{cha:chapter5_PrototypingPerf}, the LAN9514 and CM4 high-speed circuits have to be redesigned. Regarding the CM4, in Section~\ref{sec:3234_LAN9514}, the lack of the implementation of the upstream USB PHY block is addressed. It is concluded that such block should eventually be implemented in the new beRTK\textsuperscript{\textregistered}'s system, since it is necessary for the CM4 to be able to solidly toggle between host and device. Figures~\ref{fig:new_CM4_HighSpeed} and~\ref{fig:NEW_USB_Hub_1} in Annex~\ref{ann:NEW_LAN9514} show a proposed solution for the new CM4 high-speed and LAN9514 hub circuits. An USB upstream network that connects both can be seen. It must be noted that this new implementation is inspired by the design files for the CM4IO board, made available by the Raspberry Pi Foundation, as mentioned in the end of Section~\ref{sec:322_CONTROL}.

As for the USB port power controller, the one use for this project -- i.e. the MIC2026A-2YM -- could, just in case, be replaced with the active-high enable input and active-low, open-drain style fault indicator solution: the MIC2026-1BM, as proposed by~\cite{LAN9514_ref_schematic}. If that is done, the four single inverter gates (74AHC1G04) must be removed.

Looking back at Figure~\ref{fig:7_consumo}, it can be seen that, even though the CM4 seems to be in activity, all four USB downstream ports' LEDs are not lit. The unlit state prevailed even after installing the operating system in the CM4. Solving the USB data transfer problem, the LEDs will serve their original function: to indicate the power state of each USB downstream port.

Finally, it should be noted that, in order to prevent a possibly serious noise issue on the LAN9514 powering, the bypass and decoupling capacitors placed for each power pin should be relocated as close as possible to these, which can be translated as ``around the IC'', as suggested by the LAN9514 hardware design checklist manual~\cite{LAN9514_HW_Design_Checklist}.

As a side note for a future beRTK\textsuperscript{\textregistered} version based on the one developed for this project, once all the alterations are complete and the prototype fully tested, a thorough certification of all the requirements could culminate in the construction of a compliance matrix, in order to further solidify the device's reliability.




% Agricultural work, surveillance, mapping, etc. performed using robotic devices always brings many advantages, taking into account that the yield obtained turns out to be equal or (often) superior to that of a human being.
% Thus, in order to develop a device capable of doing so, the first priority will be the study of all the relevant information available, as well as all the technology known on the subject -- the study of the most relevant GNSS positioning technologies covered in Chapter~\ref{cha:chapter2_SotA} covers this aspect, corresponding to the state of the art.
% Having done this, the next step is related to the establishment of requirements to be fulfilled for the system to be developed, so that the solution obtained fulfills all the purposes initially intended -- Section~\ref{sec:III_requirements}. To this end, assimilating a chronological order made up of phases allows an estimate of the interdependencies of each one of them, as well as defining expected deadlines (Tables~\ref{tab:workflow} and~\ref{tab:gantt}).\\

% \par Successfully accomplishing complicated, hard, and time-consuming tasks is, without a doubt, something that anyone can be proud of. Nonetheless, envisioning and developing a solution capable of performing them as efficiently as possible is undoubtedly the approach that characterizes an engineer. In addition to providing valuable insights into the known literature on navigation through GNSS services -- most emphatically precise positioning --, this document also presents an estimate of the approach to be taken for the development of the project, as well as the chronological order that is considered the wisest for such.

%           PODE SER USADO NA CONCLUSAO: about RTK
% "Real Time Kinematic technique requires 2 receivers. One of them is stationary and is called "base station", the other one is "rover". The base station measures errors, and knowing that it is stationary transmits corrections to the rover (refer to How RTK works for more information about RTK). Sometimes CORS and NTRIP networks take the place of traditional base stations. They provide accurate absolute position and send corrections over the Internet. Typically the distance between the reference station and local rover shouldn't exceed 10-15 km due to the ionospheric effect. So if the reference station is located too far or simply is absent in the area you will need a local base station. Other advantages of your own base are independence from the Internet connection and lack of NTRIP subscription fees."
% - If the accurate absolute position of the base has been determined only after the job has been done, the offset of the map can be determined and corrected.

% 1. Calcular para uma charge current de 1020mA e nao de 4000mA -- fazer as contas no excel.
% 1.1 Initially the expected type of Li-ion cell to work with would be capable of being charged with a charging current of $4,000$mA, however, since the circuit ultimately did not end up with a power consumption greater than 250mA and was never powered by the backup pack for a long time, the cells never needed the full programmed bulk charge current (concept explained further ahead in this section). Therefore, due to practicality concerns,

% 2. para ter 2-level charging: instalar comparador LM311 e seguir o que diz aqui: https://www.electro-tech-online.com/threads/simple-circuit-to-invert-dc-from-positive-to-negative.141411/

% 2. This results in a value of R15 of approximately 5.36k$\Omega$, however, at the time of soldering, the closest standard resistor value available was 6.2k$\Omega$, which, when applying the same previous law to a variation of the equivalent circuit of
% % Figure~\ref{fig:voltage_divider_1}
% results in a voltage between R10 and R15 of approximately 1.15V -- sufficient to drive Q3.




% 6. The port is capable of being used as a true USB On-The-Go (OTG) port. While there is no official documentation, some users have had success making this work. The USB\_OTG\_ID pin is used to select between USB host and device that is typically wired to the ID pin of a Micro-USB connector. To use this functionality it must be enabled in the OS. If using either as a fixed slave or fixed master, please tie the USB\_OTG\_ID pin to ground. tentei isto e nao deu

% 7. As mentioned in Section~\ref{sec:3221_CM4_GPIO}, to select one of the alternative functions of a GPIO pin, the GPFSEL0 resgister's address would have to be... -->
%     for FSEL9: 011
%     for FSEL8: 011
%     for FSEL4: 011
% why? Because the ZED-F9P pins connected to the CM4's GPIO pins are part of a UART peripheral set on the ZED-F9P module, as it is possible to observe in the circuit of Figure~\ref{fig:ZED-F9P_carrier_board}, along with Section 3.1 of~\cite{ZED_F9P}.



% 10.
% Poder-se-ao colocar mounting holes na main board tambem, de forma a fixa-la na sua casing.

% 11. %ZED-F9P:
% % % ACRESCENTAR? --> para uma versão futura, pode-se usar o GNSS module desenhado pela Beyond Vision:
% Beyond Vision has developed this carrier board for an earlier project, and it can be adapted to be used in the circuit of Figure~\ref{fig:ZEDF9P_circuit} as the MOD2 GNSS module.

% Dizer isto no final:

