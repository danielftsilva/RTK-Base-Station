%!TEX root = ../template.tex
%%%%%%%%%%%%%%%%%%%%%%%%%%%%%%%%%%%%%%%%%%%%%%%%%%%%%%%%%%%%%%%%%%%%
%% chapter4.tex
%% NOVA thesis document file
%%
%% Chapter with lots of dummy text
%%%%%%%%%%%%%%%%%%%%%%%%%%%%%%%%%%%%%%%%%%%%%%%%%%%%%%%%%%%%%%%%%%%%

\typeout{NT FILE chapter4.tex}

\chapter{Conclusion}\label{cha:IV_conclusion}

Considering the impact of robotic devices in the most diverse types of fields (whether scientific or not), their precise development is of great importance, so the definition and implementation of a project in an orderly and careful way is quite valuable.

Agricultural work, surveillance, mapping, etc. performed using robotic devices always brings many advantages, taking into account that the yield obtained turns out to be equal or (often) superior to that of a human being.
Thus, in order to develop a device capable of doing so, the first priority will be the study of all the relevant information available, as well as all the technology known on the subject -- the study of the most relevant GNSS positioning technologies covered in Chapter~\ref{cha:II_SotA} covers this aspect, corresponding to the state of the art.
Having done this, the next step is related to the establishment of requirements to be fulfilled for the system to be developed, so that the solution obtained fulfills all the purposes initially intended -- Section~\ref{sec:III_requirements}. To this end, assimilating a chronological order made up of phases allows an estimate of the interdependencies of each one of them, as well as defining expected deadlines (Tables~\ref{tab:workflow} and~\ref{tab:gantt}).\\

\par Successfully accomplishing complicated, hard, and time-consuming tasks is, without a doubt, something that anyone can be proud of. Nonetheless, envisioning and developing a solution capable of performing them as efficiently as possible is undoubtedly the approach that characterizes an engineer. In addition to providing valuable insights into the known literature on navigation through GNSS services -- most emphatically precise positioning --, this document also presents an estimate of the approach to be taken for the development of the project, as well as the chronological order that is considered the wisest for such.

