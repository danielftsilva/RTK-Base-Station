%!TEX root = ../template.tex
%%%%%%%%%%%%%%%%%%%%%%%%%%%%%%%%%%%%%%%%%%%%%%%%%%%%%%%%%%%%%%%%%%%%
%% abstrac-en.tex
%% NOVA thesis document file
%%
%% Abstract in English([^%]*)
%%%%%%%%%%%%%%%%%%%%%%%%%%%%%%%%%%%%%%%%%%%%%%%%%%%%%%%%%%%%%%%%%%%%

\typeout{NT FILE abstrac-en.tex}

% Regardless of the language in which the dissertation is written, a summary is required in the same language as the main text and another summary in another language. It is assumed that the two languages in question are Portuguese and English.

% The abstracts should appear first in the language of the main text and then in the other language. For example, if the dissertation is written in Portuguese the abstract in Portuguese will appear first, then the abstract in English, followed by the main text in Portuguese. If the dissertation is written in English, the abstract in English will appear first, then the abstract in Portuguese, followed by the main text in English. 

% In the \LaTeX\ version, the NOVAthesis template will automatically order the two abstracts taking into account the language of the main text. You may change this behaviour by adding
% \begin{verbatim}
%     \abstractorder(<MAIN_LANG>):={<LANG_1>,...,<LANG_N>}
% \end{verbatim}
% \noindent to the customization area in the document preamble, e.g.,
% \begin{verbatim}
%     \abstractorder(de):={de,en,it}
% \end{verbatim}

% The abstracts should not exceed one page and, in a generic way, should answer the following questions (it is essential to adapt to the usual practices of your scientific area):

% \begin{enumerate}
%   \item What is the problem?
%   \item Why is this problem interesting/challenging?
%   \item What is the proposed approach/solution?
%   \item What results (implications/consequences) from the solution?
% \end{enumerate}

The everlasting interest in the development of devices able to perform the most various tasks has brought unquestionable advatages to an increasingly automated world. UAVs represent really well this type of devices, most prominently in agriculture. However, the time-consuming need to constantly control these devices only removes the task's physical labour factor, since the typically known GNSS positioning methods are not realiable for real-time applications. The solution for this lies on designing another device capable of performing position corrections in such conditions, commonly known as a base station.

Over the years, the development of precise positioning techniques has been perfected, with examples such as Post-Processed Kinematic (PPK) or Real-Time Kinematic (RTK) being some of the most popular ones; the latter -- as the name implies -- is the definite choice for applications that require zero latency feedback for navigation dynamics.

This document provides an exploration of the current navigational methods' state of the art and work expectations for a future Dissertation aiming to design and implement the new version of the RTK base station developed by the company Beyond Vision -- known as beRTK\textsuperscript{\textregistered}. The project will be focused on lowering the system's overall power consumption through the improvement of its general architecture.
% se perguntarem porque e que meti aqui "RTK base station" e la em cima so meti "base station": porque a base station da BV nao faz PPK positioning, enquanto que ha outras base stations que fazem, por isso, "base station" é termo geral. Isto é backed-up pelo facto de que, no site da BV, a base station é designada por "RTK base station".

% Palavras-chave do resumo em Inglês
\begin{keywords}
  GNSS, UAV, base station, precise navigation, real-time kinematic (RTK) positioning, network RTK 
  %automatic, base station, UAV, GNSS, GPS, DGNSS, RTK, PPK, PPP, accuracy, precision 
  % ver order que meti no relatorio
\end{keywords} 
