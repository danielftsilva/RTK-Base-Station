%!TEX root = ../template.tex
%%%%%%%%%%%%%%%%%%%%%%%%%%%%%%%%%%%%%%%%%%%%%%%%%%%%%%%%%%%%%%%%%%%%
%% chapter2.tex
%% NOVA thesis document file
%%
%% Chapter with the template manual
%%%%%%%%%%%%%%%%%%%%%%%%%%%%%%%%%%%%%%%%%%%%%%%%%%%%%%%%%%%%%%%%%%%%

\typeout{NT FILE chapter2.tex}

% \printbibliography[heading=subbibliography, segment=\therefsegment, title={\bibname\ for chapter~\thechapter}]
\glsresetall
\chapter{State of the Art}\label{cha:II_SotA}
% escrever aqui qlq coisa de introduçao:

Like any project development, it is always a wise approach studying all the theoretical concepts and current state of the art beforehand. That's the objective of this Chapter;
starting with the clarification of the intricate arrangement of the satellites that constitute a GNSS in section~\ref{sec:II_gnss}, the different positioning techniques inevitably become easier to dissect, through sections~\ref{sec:II_ppp},~\ref{sec:II_rtk},~\ref{sec:II_ppk} and~\ref{sec:II_ntrip}.
Section~\ref{sec:II_battery} then details the battery system of the device, as well as possible solutions to reduce the overall power consumption of this unit (and therefore of the entire equipment).
Finally, the current most relevant solutions (base stations) are addressed and compared, as these will provide helpful guidelines to the development of future beRTK.

% enter a NUTSHELL setting of RTK,
a base station is intended that is capable of fine-tuning the position of the rover (drone);

---------------------

beRTK can be used for:
Enhanced Landing
Surveying fields precisely
Accurate Mapping
Review Work Processes

HEIFU, beRTK and beXStream can be used for:
\begin{itemize}
    \item \textbf{Cartografy and 3D Mapping}
    \item \textbf{Detection and Removal of Asian Wasp hives}
    \item \textbf{Precision Farming}
        a. Survey every inch of your field
            Allows detailed survey of every field
        b. Improves productivity and efficiency 
            Reduce time spent on field-walking
        c. Higher resolution of the field areas of interest 
            It is beneficial for immediate identification and GPS tagging
        d. Earlier identification of potential crop issues
            Use of multispectral sensing technology
    \item \textbf{Search and Rescue}
    \item \textbf{Forest Firing Detection and Monitoring}
    \item \textbf{Border Patrol Monitoring}
    \item \textbf{Demining}
    \item \textbf{Medical Delivery}
    \item \textbf{Infrastructure}
    \item \textbf{High Rise Window Cleaning}
    \item \textbf{Ad hoc Communication Relay}
    \item \textbf{Equipment Monituring in factories}
    \item \textbf{Remote Car Inspection - Insurance}
    \item \textbf{Virtual Traveling}
\end{itemize}
\begin{itemize}
    \item But there is still a great ally for such efficiency, which sometimes goes unnoticed or is even forgotten: Image Processing.

    % Começar com:
    % I. o que é uma base station? fazer analogia
    % II. dizer para que e que serve
    % III. onde/no que é que eu vou empregá-la?
    % IV. que tecnologias utiliza?
    % V. explicar cada tecnologia relativamente a base

    % artigos que li:
    \item a. Experimental Testbed and Methodology for the Assessment of RTK GNSS Receivers Used in Precision Agriculture;

    \item b. DETERMINATION OF THE POSITION USING RECEIVERS INSTALLED IN UAV

    \item c. High-Precision/Throughput Growth Measurement of Crops by Drone with Stereo Matching Based on RTK-GNSS and Single Camera

    \item d. Estimation of the Base Station Position Error in a RTK Receiver Using State Augmentation in a Kalman Filter

    \item e. Resilient Deployment of Drone Base Stations

    \item f. Based on a single-base station RTK control survey and precision analysis 

    \item g. Design of an Autonomous drone for IoT deployment analysis 

    \item h. RTK+ System for Precise Navigation in Shadowed Areas 
\end{itemize}

\section{Global Navigation Satellite System}\label{sec:II_gnss}

Whenever someone wishes to know their current location on Earth, just a few, effortless taps on a smartphone will be the quickest way to do it; this is often associated with the radionavigational system of GPS, which has been around for many years. In a more general manner, this technology can be described as a Global Navigation Satellite System, or GNSS, which refers to any satellite constellation that can be used in order to help navigation throughout the world (as the name suggests).
Therefore, it is possible to conclude that, in order to help tracking their location through the use of satellites, typical smartphones or GPS navigation systems in cars have GNSS receivers, just like a specially designed surveying device.

According to~\cite{novatel_gnss}, the best way to address GNSS receivers as a whole is to start by acknowledging its working basis: satellites. While orbiting the Earth, these machines send out signals that are then acknowledged by receivers (hence the name), assisting them on the calculation of their own location on Earth by comparing the received information from other satellites.

% minha parte: METER ISTO NA INTRODUÇÃO DESTA SECÇÃO - GNSS (GERAL)
Through~\cite{fed_rad_plan_2008}, it is possible to understand that, in order to enjoy the perks of assisted navigation, the quality behaviour of a GNSS must be ensured. 
A smart approach for that is to analyse the following four performance criteria:

\begin{itemize}
    \item Accuracy: Should not be confused with precision; this parameter acts as a measure of coherence between the estimated (predicted) and actual position of a vehicle, aircraft, or vessel, at any given time;
    \item Availability: In a percentile manner, this gives the navigator the ability to know the amount of time available to benefit from the services provided by the system, within a certain specified coverage area.
    \item Continuity: Refers, in probabilistic terms, to the ability of the entire system of maintaining all required functions, whilst keeping any interruptions out of question. This ensures the operation of the system in a smooth way within a given time interval. All of this assuming that the system was fully available at the start of this operational phase;
    \item Integrity: Measuring the trustworthiness of the information supplied by a navigation system, integrity is also able to consistently issue warnings when navigation through the use of the system is not recommended.
\end{itemize}
All of these parameters are defined as a way to rate the performance of a satellite constellation, which helps tracking down any discrepancies or even problems that might make accurate navigation unacceptable.
Their foundation is known as the Required Navigation Performance (RNP) specification, which characterizes the imperative performance indicators within a defined airspace. By ensuring the proper conduct of a constellation, the perks of a GNSS for navigation purposes are clear.

Therefore, it is possible to understand that a GNSS comprises a network of satellites that, while continuously orbiting the planet, constantly emit radiofrequency signals carrying information about their current status, position in space and precise time.
This information is achieved through atomic clocks, installed within the satellite itself. Thus, the process unfolds as follows:

\begin{enumerate}
    \item Satellites will start the transmission of their position in real time;
    \item As this happens, the receiver on Earth will be looking for a signal from the satellite, and by the time that signal is received, there will be a delay between transmission and reception, beign that radio signals travel at the speed of light ($c$);
    \item Knowing both these timestamps, a GNSS receiver is then able to calculate the difference between the two and determine the time it took to receive the signal;
    \item Lastly, multiplying this calculated time interval by the speed of light, it is then possible to find the distance from the satellite to the receiver.
\end{enumerate}
This process is known as ``trilateration'' and is explained in more detail in the following section.

\subsection{Trilateration}\label{sec:II_gnss_trilateration}

Incorporating the idea that the time ($t$) it takes to receive a signal from a satellite multiplied by the speed of light ($c$) will equal the distance ($d$),

\begin{equation}\label{eq:2_1}
    d = v\,t = c\,t\,,\medskip
\end{equation}
it indicates that the signal emitted by the satellite propagates in an omnidirectional way, which can be pictured as a sphere around the satellite -- as it's known from radiation and propagation of electromagnetic waves theory --, meaning the signal will reach the Earth in numerous locations (Figure~\ref{fig:omnidirectional}).
Having another satellite orbiting around Earth, the signal emitted by it will reach the surface at some other particular time. Geometrically, this means that the intersection of the spheres that represent these two signals will correspond to a circle, which limits the extensive list of possible solutions -- note that, however, there is still a large amount --, not allowing yet to have an exact location. Adding a third satellite to this scenario will further limit the possible locations of the receiver, narrowing them down to only two points: one in space and another one down on the surface of the Earth. Knowing that a receiver is down on Earth, by using the latter as a fourth surface, the correct location (in a set of $x$, $y$, $z$ coordinates) can be determined.
% meter imagem word: propagação
\begin{figure}[ht]
	\centering
	\includegraphics[width=1.0\textwidth]{Chapters/Figures/demo.png}
	\caption{Propagation of satellite signals.}
	\label{fig:omnidirectional}
\end{figure}
However, just three satellites are not enough for an accurate reading, although in theory this process should be enough. The practical use of trilateration must account with a minimum of four satellites in direct line of sight of any point on the surface of the Earth, in order to synchronise the receiver's clock with the satellites', as well as to better the precision of the solution.
There is, nonetheless, one particular phenomenon able to damage the received signal, resulting in less accurate calculations; it is called Dilution of Precision (DOP). This numerical quantity can be measured statistically, while accounting for satellite geometry and location (relative to the receiver) and may, for instance, be impacted by atmospheric or even urban factors.
% meter imagem de predios e sinais afetados
Logically, it is immediately (and rightfully) assumed that the more satellites there are, the higher the positional accuracy will be for the receiver. For that, to achieve a successful observation, three major segments are necessary, known as the space, control and user segments~\cite{gps_USGov,novatel_gnss,ayers_geosystems_2011}, as depicted by Figure~\ref{fig:s_c_u_segment}.

\begin{figure}[ht]
	\centering
	\includegraphics[width=1.0\textwidth]{Chapters/Figures/demo.png}
	\caption{GNSS segments.}
	\label{fig:s_c_u_segment}
\end{figure}

\subsection{Space Segment}\label{sec:II_gnss_space_seg}

The space segment corresponds to the GNSS satellites. All of these follow a certain orbital path thousands of kilometers above the Earth's surface, aiming to assist in the precise location of GNSS-enabled devices on the ground.
Different nations around the world have various satellite networks, with GPS being the most widely known GNSS of all. Developed and established by the United Satates Department of Defense near the end of the 1970s, GPS relies on a constellation made up of a total of 31 satellites, on which the U.S. government works to maintain constant operability of a minimum of 24, 95\% of the time, to guarantee global coverage~\cite{gps_USGov}.
Upon its the introduction to civilian use, the desire to cover an even larger portion of the globe only grew, which led to the development of other GNSSs that, together with the already existing GPS, could improve navigational accuracy. Thus, there are currently four global satellite arrangements that make world navigation and precise positioning possible:

% table:-----------------------------
\begingroup
\begin{table}[h]
	\caption{Global satellite positioning systems.}
	\label{tab:5_GNSSs}
	\centering%@{}l@{}@{}c@{}@{}c@{}@{}c@{}@{}c@{}
    % \setlength{\tabcolsep}{10pt} % Default value: 6pt
    % \renewcommand{\arraystretch}{1.5} % Default value: 1
	\begin{tabular}{lcccccc}
        \toprule
        \multicolumn{2}{c}{\multirow{2}*{\textbf{Constellation}}} & \multirow{2}*{\textbf{Satellites}} & \textbf{Orbital} & \textbf{Orbital}     & \textbf{Orbit} \\
        \multicolumn{2}{c}{}                                      &                                    & \textbf{Planes}  & \textbf{Inclination} & \textbf{Radius} \\
        \midrule
     
        \multirow{3}*{\includegraphics[height=1cm]{Chapters/Figures/flags/usa.png}}&\multirow{3}*{GPS} &  &  &  & \\
        \multirow{3}*{}   &{}             & 27 + 4 (spares) & 6 & 55 degrees & $20,200$ km \\
        \multirow{3}*{}   &{}          & & & & \\

        \midrule

        \multirow{3}*{\includegraphics[height=1cm]{Chapters/Figures/flags/Russia.png}}&\multirow{3}*{GLONASS} &  &  &  & \\
        \multirow{3}*{}   &{}             & 24 + 3 (spares) & 3 & 64.8 degrees & $19,140$ km \\
        \multirow{3}*{}   &{}          & & & & \\

        \midrule

        \multirow{3}*{\includegraphics[height=1cm]{Chapters/Figures/flags/Europe.png}}&\multirow{3}*{Galileo} &  &  &  & \\
        \multirow{3}*{}   &{}             & 27 + 3 (spares) & 3 & 56 degrees & $23,222$ km \\
        \multirow{3}*{}   &{}          & & & & \\

        \midrule
        \multirow{3}*{\includegraphics[height=1cm]{Chapters/Figures/flags/China.png}}&\multirow{3}*{BeiDou} & 5 GEO & --- & --- & $35,787$ km \\
        \multirow{3}*{}   &{}             & 3 IGSO & --- & 55 degrees & $35,787$ km \\
        \multirow{3}*{}   &{}          & 27 MEO & 3 & 55 degrees & $21,525$ km \\
        \bottomrule
    \end{tabular}
\end{table}
\endgroup

%meter/FAZER!!! polygons - imagens das coverages:    !!!!!
% !!!!!!!!!!!!!!!!!!!!!!!!!!!!!!!!!!!!!!!!!!!!!!!!!!!!!!!!

On an additional manner, there are also two extra systems that were developed to provide regional coverage:

\begin{itemize}
    \item NavIC (India): Planned to be made up of seven satellites, this Regional Navigation Satellite System was developed by India to grant regional coverage (hence the (outdated) name IRNSS) to India and its neighbouring area.
    As a result of the launch of the constellation's last satellite in 2016, the IRNSS title was changed to Navigation Indian Constellation (NavIC), by India's Prime Minister Narendra Modi~\cite{navic_news_2016};
    \item QZSS (Japan): Working as another regional navigation satellite system, the development of the Quasi-Zenith Satellite System was accredited by the Japanese government in order to ensure service to Japan, as well as the Asia-Oceania region and consists of four satellites.
    However, this constellation limits accuracy in its standalone (also known as single-point positioning) mode, which means it functions as GPS augmentation service\footnote{Augmentation services will be covered in more detail in section~\ref{sec:II_gnss_sbas}.} by synchronisation of clocks use of same frequencies, allowing the use of this system's machinery as extra GPS satellites.
\end{itemize}

Every constellation follows the same purpose of providing services of:

\begin{itemize}
    \item Positioning: Refers to the ability of determining one's location and orientation in an accurate and precise manner, either two-dimensionally or three-dimensionally (whenever necessary), taking a standard reference geodetic system (e.g. World Geodetic System 1984, or WGS84);
    \item Navigation: Enables the determination of the current position, as well as the application of course, orientation and velocity corrections in order to achieve a desired position (relative or absolute) at any location in the world, from subsurface to surface and from surface to space;
    \item Timing: Also encompassing time transfer, timing refers to the ability of acquiring the exact and precise time from a certain standard (in this case Coordinated Universal Time, or UTC), as well as maintain it, anywhere in the world and within timeliness parameters defined by the user.
\end{itemize}
These are known as PNT services~\cite{pnt_2017}.
A good example of the use of this type of data is the GPS system, which is a result of the use of PNT data together with information from maps, as well as from other sources (e.g. meteorological data, traffic, etc.).

Therefore, the space segment essentialy focuses on accurately covering every point on Earth's surface, something which engineers work on to improve everyday, whether it being with precision tweaks to the satellites' atomic clocks or even launching new ones to further complement the existing GNSSs~\cite{euspa_news_2022}. This means that all constellations complement each other, and by being used globally, the location accuracy at any given time is increased.

\subsection{Control Segment}\label{sec:II_gnss_control_seg}

Each satellite's status -- more specifically, its ``health'', position (current and expected), velocity and timing -- is recorded by an internal orbital pattern known as \gls{ephemeris}. Such data is, in turn, included within the transmitted signals. The control segment is found on Earth, at a stationary location, and its control element is used to correct any errors coming from the satellites, being that the more satellites that can be observed in the control segment, the greater the probability of an error to be detected and corrected, thus increasing accuracy in the user segment.

A good example for the control segment is, precisely, a base station: by installing a GNSS receiver at a fixed location, it will continuously collect data from the satellites, as well as any changes or discrepancies in the readings, which can then be corrected later and transferred to the user segment. Nowadays, there are long-term reference stations that are mounted and freely accessible, which eliminates the need for two receivers; having the base and rover together can be too much equipment in some situations and therefore result in larger financial costs.
So, in order to solve this, the rover receiver just needs to be connected to some kind of frame of reference, which will be the control segment used to complete the solution (this could be, for example, an NTRIP network\footnote{NTRIP will be covered in more detail in section~\ref{sec:II_ntrip}.}).
This type of references are generally found in heavily populated areas; anywhere that is more rural or not really connected to society, they may not be available, so the need to have two receivers (base station and rover) in order to complete the three segments will inevitably be felt.

One of the tasks of this segment that is worth mentioning is the constant monitoring of the satellites, i.e. the regular adjustment of trajectory and time information, in order to keep all the transmitted information highly accurate.

\subsection{User Segment}\label{sec:II_gnss_user_seg}

The user segment includes smartphones, rovers and anything that has a GNSS receiver installed; Basically, it is the end result and refers to what needs to be measured (i.e. the location to find). The other two segments act only as a means of helping reach a more precise position for the user segment.
As we try to find the location of a point in a given \gls{geodetic_datum} or coordinate system, regardless of what is being done, taking a survey is about ensuring that all data collected is accurate and any correction elements used are fully understood.

\subsection{Satellite Communication}\label{sec:II_gnss_comm}

It has already been affirmed in section~\ref{sec:II_gnss_trilateration} that GNSS satellites communicate with receivers through radiofrequency signals -- which are electromagnetic signals -- that propagate omnidirectionally. As the name states, these signals rely rigourously on frequency, and that leaves the wonder on how satellite communication actually works.

These machines are designed to continuously orbit along their respective trajectories, and therefore the signals transmitted might run into problems while on their way to Earth, which can occur due to atmospheric reasons or even while the receiver is blocked by something such as a building, for example. An affected signal reaching a recevier will result in a poor reading precision. Table~\ref{tab:GNSS_sys_errors} shows the typical errors intrinsic to the GNSS system.

% typical errors: ATENÇÃO À POSIÇÃO DA TABELA!!
\begingroup
\begin{table}[h]
	\caption{Typical GNSS system errors (adapted from~\cite{novatel_gnss}).}
	\label{tab:GNSS_sys_errors}
	\centering%@{}l@{}@{}c@{}@{}c@{}@{}c@{}@{}c@{}
    % \setlength{\tabcolsep}{10pt} % Default value: 6pt
    % \renewcommand{\arraystretch}{1.5} % Default value: 1
	\begin{tabular}{lc}
        \toprule
        \textbf{Source} & \textbf{Error range} \\
        \midrule     
        Satellite clocks & $\pm 2.0$ m \\
        \midrule
        Orbit errors & $\pm 2.5$ m \\
        \midrule
        Ionospheric delays & $\pm 5.0$ m \\
        \midrule
        Tropospheric delays & $\pm 0.5$ m \\
        \midrule
        Receiver noise & $\pm 0.3$ m \\
        \midrule
        Multipath & $\pm 1.0$ m \\
        \bottomrule
    \end{tabular}
\end{table}
\endgroup

The communication process can be characterized through an intricate set of terms such as \gls{pseudorandom}, \gls{correlation} and code division multiple access (CDMA). According to~\cite{novatel_gnss}, this categorizes the GNSS positioning technique as ``code-based'', meaning the receiver operates through pseudorandom noise codes (or PRN codes) transmitted by four or more satellites, correlating with these in order to determine time and position. GPS is an excellent example capable of clarifying these concepts.

% dar o exemplo de GPS: tem CDMA -> falar de CDMA:
The way GPS satellites communicate through the same frequency bands is through CDMA. This form of spread spectrum enables the modulation of signals via a code defined as being pseudorandom, meaning that receivers can, beforehand, associate the code intrinsic to each satellite, which allows synchronisation (correlation) between themselves and that same CDMA code. That way, theoretically, information should be able to flow smoothly, however, due to the aforementioned dilution of precision (DOP), signals may be affected while travelling through the atmosphere;
the propagation signal can be affected when it passes through the ionosphere (a layer located roughly 50-1,000 km above Earth's surface) or the troposphere (located about 8-14.5 km above the surface), which correspond to the upper and lower zones of the atmosphere, respectively. These effects depend significantly on frequency, so it is worth noting that signals can be compromised by negative effects such as reflection and even absorption, scintillation, Faraday rotation and \textbf{bandwidth decoherence}~\cite{au_gov_satell}. Considering these problems, GPS was designed to operate in frequencies within the interval of 1 to 2 GHz, which corresponds to a section of the radio spectrum known as the L-band, as represented by Figure~\ref{fig:gps_bands} (which shows the frequncy bands used in GPS communication, known as the L1, L2 and L5 frequencies).
% meter/fazer figura das frequencias do GPS:
\begin{figure}[ht]
	\centering
	\includegraphics[width=1.0\textwidth]{Chapters/Figures/demo.png}
	\caption{GPS frequency bands.}
	\label{fig:gps_bands}
\end{figure}

This band solves a great part of the stated problems due to its high frequencies, as it 

% razões 1: MUDAR
Simplification of antenna design. If the frequency had been much higher, user antennas may have had to be a bit more complex.
- Ionospheric delay is more significant at lower frequencies.
- Except through a vacuum, the speed of light is lower at lower frequencies
- The coding scheme requires a high bandwidth, which was not available in every frequency band.
- The frequency band was chosen to minimize the effect that weather has on GPS signal propagation.

% razões 2: MUDAR
L1 transmits a navigation message, the coarse acquisition C/A code (freely available to the public) and an encrypted precision (P) code, called the P(Y) code (restricted access). The navigation message is a low bit rate message that includes the following information:
- GPS date and time.
- Satellite status and health. If the satellite is having problems or its orbit is being adjusted, it will not be usable. When this happens, the satellite will transmit the out-of-service message.
- Satellite ephemeris data, which allows the receiver to calculate the satellite's position. This information is accurate to many, many decimal places. Receivers can determine exactly where the satellite was when it transmitted its time.
- Almanac, which contains information and status for all GPS satellites, so receivers know which satellites are available for tracking. On start up, a receiver will recover this "almanac".
The almanac consists of coarse orbit and status information for each satellite in the constellation.

The P(Y) code is for military use.

GPS. The L2 frequency transmits the P(Y) code and, on newer GPS satellites, it also transmits the C/A code (referred to as L2C), providing a second publicly available code to civilian users.

\subsubsection{Signal Propagation}\label{sec:II_gnss_comm_propag}

The layer of the atmosphere that most influences the transmission of GPS (and other GNSS) signals is the ionosphere. These electrons influence electromagnetic wave propagation, including GPS satellite signal broadcasts. Ionospheric delays are frequency dependent so by calculating the range using both L1 and L2, the effect of the ionosphere can be virtually eliminated by the receiver.

The other layer of the atmosphere that influences the transmission of GPS signals is the troposphere. Tropospheric delay is a function of local temperature, pressure and relative humidity. L1 and L2 are equally delayed, so the effect of tropospheric delay cannot be eliminated the way ionospheric delay can be.

Some of the signal energy transmitted by the satellite is reflected on the way to the receiver. This phenomenon is referred to as "multipath propagation" and is represented in Figure~\ref{fig:multipath}. These reflected signals are delayed from the direct signal and, if they are strong enough, can interfere with the desired signal. Techniques have been developed whereby the receiver only considers the earliest-arriving signals and ignores multipath signals.

%meter imagem de sinais de satelite a serem deflected e refleced em predios e tal:
\begin{figure}[ht]
	\centering
	\includegraphics[width=1.0\textwidth]{Chapters/Figures/demo.png}
	\caption{GNSS satellite signal multipath propagation.}
	\label{fig:multipath}
\end{figure}

% example of multipath propagation:--------------------------------------
As an example of the effects of mulipath propagation:
If either the transmitter or the receiver, or both, are inside man-made structures, then additional propagation losses typically occur. These losses should be added (in decibels) to the propagation losses computed using the models described previously.
Building penetration losses vary considerably with building construction, materials, structure, and the location of the receiver or transmitter within the building. The presence or absence of windows, and even the difference between metal window frames and wooden window frames, can make a significant difference in the propagation loss into a room. Besides propagation loss, signals received within a building often experience significant multipath. 
Building penetration losses are discussed extensively [41]. The following discussion is based on [40]. The excess loss in decibels due to building penetration is typically modeled, for signals arriving from above the building, as

\begin{equation}
    L=L_{roof}+n_{floor}\,L_{floor}\,,\medskip
\end{equation}
where
- Lroof is the roof penetration loss, which can range from 1 dB to 30 dB at L-band;
- nfloor is the number of floors penetrated;
- Lfloor is the loss per floor, which can range from 1 dB to 10 dB at L-band. For building penetration through walls, the excess loss in decibels is similarly typically modeled as

\begin{equation}
    L=L_{ext}+n_{int}\,L_{int}\,,\medskip
\end{equation}
where
- Lext is the exterior wall penetration loss, which can range from 1 dB to 30 dB at L-band;
- nint is the number of interior walls penetrated;
- Lint is the loss per interior wall, which can range from 1 dB to 10 dB at L-band.

Table 9.13 lists representative losses for different building materials, drawing from an extensive set of measurements reported in [42].

%meter tabela 9.13

%----------------------------------------------------------------------

Upon reception of the signals:
The use of more satellites, if they are available, will improve the position solution;

To determine a fix (position) and time, GNSS receivers need to be able to track at least four satellites. This means there needs to be a line of sight between the receiver's antenna and the four satellites.

Receivers vary in terms of which constellation or constellations they track, and how many satellites they track simultaneously. For each satellite being tracked, the receiver determines the propagation time. It can do this because of the pseudorandom nature of the signals.

Since the receiver knows the pseudorandom code for each satellite, it can determine the time it received the code from a particular satellite. In this way, it can determine the time of propagation.

% fim da parte de propagação.

Important requirement: the requirement of CDMA to operate in a high bandwidth, therefore truly benefiting from this L-band.

It was also specially selected to work on due to the fact that it 

C/A P-code~\cite{ca_p_code_1991}

It is worth stating that the navigation message transmitted is has a low bit rate.

This can explain the reason why these signals should be transmitted in high frequencies, which paves the ground for the next question: Which frequency bands yield the best transmission results for each GNSS?

\subsubsection{Frequency Bands}\label{sec:II_gnss_comm_freq_bands}

% no final meter tabela: ATENÇÃO À POSIÇÃO DA TABELA!!
\begingroup
\begin{table}[h]
	\caption{GNSS signals and frequencies (adapted from~\cite{novatel_gnss}).}
	\label{tab:GNSS_freqs}
	\centering%@{}l@{}@{}c@{}@{}c@{}@{}c@{}@{}c@{}
    % \setlength{\tabcolsep}{10pt} % Default value: 6pt
    % \renewcommand{\arraystretch}{1.5} % Default value: 1
	\begin{tabular}{lcc}
        \toprule

        \textbf{System} & \textbf{Signal} & \textbf{Frequency (MHz)} \\

        \midrule
     
        \multirow{5}*{GPS}      & L1 C/A    & 1575.42 \\
        \multirow{5}*{}         & L1C       & 1575.42 \\
        \multirow{5}*{}         & L2C       & 1227.60 \\
        \multirow{5}*{}         & L2P       & 1227.60 \\
        \multirow{5}*{}         & L5        & 1176.45 \\

        \midrule

        \multirow{4}*{GLONASS}  & L1 C/A    & 1598.0625-1609.3125 \\
        \multirow{4}*{}         & L2C       & 1242.9375-1251.6875 \\
        \multirow{4}*{}         & L2P       & 1242.9375-1251.6875 \\
        \multirow{4}*{}         & L3 OC     & 1202.025            \\
        
        \midrule

        \multirow{5}*{Galileo}  & E1        & 1575.42  \\
        \multirow{5}*{}         & E5a       & 1575.42  \\
        \multirow{5}*{}         & E5b       & 1207.14  \\
        \multirow{5}*{}         & E5 AltBOC & 1191.795 \\
        \multirow{5}*{}         & E6        & 1278.75  \\

        \midrule

        \multirow{6}*{BeiDou}   & B1l       & 1561.098 \\
        \multirow{6}*{}         & B2l       & 1207.14  \\
        \multirow{6}*{}         & B3l       & 1268.52  \\
        \multirow{6}*{}         & B1C       & 1575.42  \\
        \multirow{6}*{}         & B2a       & 1176.45  \\
        \multirow{6}*{}         & B2b       & 1207.14  \\

        \midrule

        NavIC     & L5        & 1176.45 \\

        \midrule

        \multirow{6}*{QZSS}     & L1C/A     & 1575.42 \\
        \multirow{6}*{}         & L1C       & 1575.42 \\
        \multirow{6}*{}         & L1S       & 1575.42 \\
        \multirow{6}*{}         & L2C       & 1227.6  \\
        \multirow{6}*{}         & L5        & 1176.45 \\
        \multirow{6}*{}         & L6        & 1278.75 \\

        \midrule

        \multirow{2}*{SBAS}     & L1        & 1575.42 \\
        \multirow{2}*{}         & L5        & 1176.45 \\

        \bottomrule
    \end{tabular}
\end{table}
\endgroup

%meter imagem das frequencias de todas aonstelações no mesmo referencial: figura 36 novatel_gnss
% meter imagem svg do planeta com todas as GNSS?

%falar de cada frequência: ...
L5:
The United States has implemented a third civil
GPS frequency (L5) at 1176.45 MHz. The modernized
GPS satellites (Block II-F and later) are
transmitting L5.
the modernied benefits of the L5 signal include meeting
the requirements for critical safety-of-life applications
such as that needed for civil aviation
and providing:
- Improved ionospheric correction.
- Signal redundancy.
- Improved signal accuracy.
- Improved interference rejection.

% NOTA FINAL SOBRE GNSS:
As GNSS constellations and satellites are added, we will be able to calculate position more accurately and in more and more places.



\section{GNSS Augmentation}\label{sec:II_gnssAug}

Regarding precise navigation, the precision obtained solely by relying on GNSS is not good enough (as it only offers meter-level accuracy~\cite{novatel_gnss}) -- possible to conclude by looking at Table~\ref{tab:GNSS_sys_errors}. Taking the example of an autonomous drone used in all the processes mentioned in section \textbf{??}
%~\ref{sec:} // falta ainda mencionar os processos dos drones!
, the need for a higher precision degree is definetely clear. To make such navigation possible, the top priority is to bring GNSS system errors down to a minimum.

At this point it is clear that the focal point behind GNSS positioning is related to equation (\ref{eq:2_1}), which means that any error connected to distance, velocity or time will affect the transmitted signal's quality, and thus, mitigation of these problems will yield better results.
The way to do so is through techniques relying on the augmentation of GNSS positioning capabilities~\cite{novatel_gnss,kaplan_2017}, which portrays an ingenious way of improving the four performance parameters described in section~\ref{sec:II_gnss} (accuracy, availability, continuity and integrity).
These techniques range from least to most effective, and can rely on:

\begin{itemize}
    \item Average calculation of recurring measurements at the same location;
    \item Prediction of correction values through modeling of the event causing the error;
    \item Differential Corrections (DGNSS).
\end{itemize}
DGNSS solutions represent the highest efficiency attainable. % citation needed

%For the purpose of this dissertation, the focus of error correction through GNSS augmentation will be set on differential corrections (more precisely, Differential GNSS, or DGNSS), as this is method relates directly with the beRTK base station.

% Breaking augmentation systems in a twofold way, accuracy and integrity ssuring further development of accuracy from GNSS augmentation 

% % wiki:
% Based on its main feature, GNSS augmentation systems can be classified as those providing integrity information to the primary GNSS satellites constellation(s) and those improving the accuracy of the user solution with respect to the only use of the primary GNSS constellation(s).

% A further classification may be done according to an additional relevant feature, which for the former relates on whether the augmentation information comes from satellites (satellite-based augmentation system) or from ground (ground-based augmentation system) and for the latter on whether the accuracy improvements use a dense network of reference stations (Differential GNSS, Real Time Kinematics - RTK - or Wide Area RTK (WARTK)) or just a few stations (Precise Point Positioning - PPP) for the computation of the augmentation information.

% meter imagem: wiki Accuracy Performances for GNSS and GNSS Augmentation Techniques - METER (adapted from) !!!!!!!!!!!!!
\begin{figure}[ht]
	\centering
	\includegraphics[width=1.0\textwidth]{Chapters/Figures/demo.png}
	\caption{Accuracy Performances for GNSS and GNSS Augmentation Techniques.}
	\label{fig:accuracy_perfromances}
\end{figure}



% --------------- PARTE A VERDE: DONE
The conventional code-based method presented in section~\ref{sec:II_gnss_comm} is not capable of providing low-order levels of precision, so another method known as ``carrier-based'' was developed.
This type of method means that the measurement of the carrier wave phase corresponds to a measurement of the distance (also known as range) between the satellite and the receiver, expressed in number of cycles of the carrier frequency~\cite{inside_GNSS} (Figure~\ref{fig:dgnss_corrections}), which results in more accurate output results, ideal for UAV missions.
Among the best-known techniques that adopt this method capable of calculating highly accurate solutions (i.e. orders of magnitude more accurate than those obtained through code-based GNSS) is Real-Time Kinematic (RTK) positioning\footnote{Addressed in more detail in section~\ref{sec:II_gnssAug_rtk}.}, which is the main focus of the beRTK base station.
%meter imagem figura 40 base station a fazer correçoes:
\begin{figure}[ht]
	\centering
	\includegraphics[width=1.0\textwidth]{Chapters/Figures/demo.png}
	\caption{Differential GNSS workings.}
	\label{fig:dgnss_corrections}
\end{figure}

Taking into account the possible signal ambiguities that may arise when using a carrier-based technique, the accuracy of the solution can only be maximized after they are resolved. This is possible through phase modulation of the carrier waves using a PRN code which, as mentioned, is capable of differentiating satellite signals, as well as making timing corrections in order to perform range calculations.
% ------------------ fim da PARTE VERDE.

\subsection{Differential GNSS}\label{sec:II_gnssAug_dgnss}

% --------------- PARTE VERMELHA: DONE
Looking again at Figure~\ref{fig:dgnss_corrections}, one of the most important devices for augmented GNSS services can clearly be seen: the base station. This element is crucial to the workings of the technique able to describe the operation of RTK, known as Differential GNSS, or DGNSS. This carrier-based method consists on the use of a fixed GNSS receiver that sits in a static and known position (obtained in a highly accurate way through conventional surveying techniques) -- base station; due to the existence of an internal GNSS receiver in it, the ranges to each GNSS satellite in line of sight are effectively determined using a code-based technique (such as the conventional GNSS technique), and the location of each of these satellites is also calculated taking into account precisely known orbit ephemerides as well as the time of each satellite.
When comparing the measured position with the calculated one (from the ranges to each satellite -- also calculated), it can be clearly assumed that the differences between both positions will be ephemeris-related, as well as due to the differences between clocks 
%relative clock offsets (?)
-- bearing in mind that, unlike a GNSS satellite, a base station does not have an atomic clock. However, delays related to the passage of the signal through the atmosphere are the most recurring and common errors.
Subsequently, in order for the corrections to be incorporated in the calculated solutions, the base station performs the task of sending the detected range errors to other receivers that also obtain their position through GNSS positioning (rovers; e.g. a UAV). This correlation allows such errors to be calibrated out.
For this, it is necessary to establish a link for the supply of data between the base station and the rover, keeping in mind the fast transmission and reception of corrections happening in real-time. Connections like these can be made through radio frequency\footnote{The LoRa protocol is good example for this; it is addressed in more detail in section~\ref{sec:II_gnssAug_rtk_LoRa}.} -- by using a frequency interval with values ranging from under 300 kHz (low frequencies, or LF) to around 1-2 GHz (corresponding to the L-band (first mentioned in section~\ref{sec:II_gnss_comm})) and above -- or the internet.

In case the need arises for a quick calculation of corrections, i.e. in real-time, both devices will need to have in their line of sight at least four GNSS satellites -- three for the determination of the position according to $x$, $y$ and $z$ (as explained in~\ref{sec:II_gnss_trilateration}) and an extra one to correct for clock time differences between satellites and receivers, commonly called time offset. Thus, it is concluded that, in order to obtain a high-precision position for the rover, a high-precision position must be determined at the base station, considering the dependence between the two. If it is assumed that, for the missions performed, the rover does not move too far from the base station, the real-time accuracy of the measurements will be high, taking into account that the signals received by each of the devices will pass through, sensibly, the same atmosphere conditions. In this way, the greater the distance possible to maintain between devices, the better.

If one wishes to put aside the real-time factor of the job, there are also other techniques capable of providing precise positioning -- in some cases, the precision values may even be higher than those obtained from RTK -- through the simple use of two GNSS receivers, which are trusted with the task of logging positioning data to a storage device. Later, the gathered data can be processed however it's intended to. Section~\ref{sec:II_gnssAug_ppk} describes an example of such method, known as Post-Processed Kinematic positioning (PPK)~\cite{novatel_gnss,kaplan_2017,groves_2008}.
% ------------------ fim da PARTE VERMELHA.

\subsection{Satellite-based Augmentation System (SBAS)}\label{sec:II_gnssAug_sbas}

% --------------- PARTE A LARANJA:
Table~\ref{tab:GNSS_freqs} introduced various frequency values for different GNSS constellations, with all of them having already been addressed, except one: indicated as ``SBAS''.

SBAS stands for Satellite Based Augmentation System and its design purpose has in mind situations where the usage and financial value of the DGNSS can be circumvented, either due to a large working area or even if the base stations are spread over a very wide area. great. It should be noted that the focus of this system has a lot to do with large-scale work areas, which are where SBAS proliferates.

Making use of geosynchronous satellites, i.e., satellites that orbit the Earth with a period equivalent to the its rotation period, these systems work as auxiliaries of the currently existing GNSS networks, as they are also able to improve signal parameters such as:
\begin{itemize}
    \item Accuracy: Possible to be improved by propagating wide area corrections for all detected GNSS range errors;
    \item Availability: In a straightforward way, signal availability is possible to improve through the transmission of range signals from each SBAS satellites;
    \item Integrity: Ensured by the SBAS network via rapid detection of satellite signal errors and subsequent sending of non-tracking alerts to the receivers. 
\end{itemize}

%meter imagem dos SBAS systems: DONE
\begin{figure}[ht]
	\centering
	\includegraphics[width=1.0\textwidth]{Chapters/Figures/SBAS.png}
	\caption{SBAS systems (currently in operation/development; adapted from~\cite{sbas_euspa_2021}).}
	\label{fig:SBAS_systems}
\end{figure}

Currently, there are several SBAS services implemented that provide extra positioning precision. Figure~\ref{fig:SBAS_systems} shows the already existing SBAS, as well as the ones in current development, which are:
\begin{itemize}
    \item Wide Area Augmentation System (WAAS): Developed by the U.S. Federal Aviation Administration (FAA);
    \item System for Differential Corrections and Monitoring (SDCM): Currently in development by the Russian Federation;
    \item European Geostationary Navigation Overlay Service (EGNOS): Developed by the European Space Agency (ESA), in cooperation with the European Commission (EC) and EUROCONTROL (European Organization for the Safety of Air Navigation). Provides coverage to the majority of the European Union (EU);
    \item BeiDou SBAS (BDSBAS): Currently in development by the People's Republic of China;
    \item GPS-aided GEO-Augmented Navigation (GAGAN): Developed by the Indian government;
    \item Michibiki Satellite Augmentation System (MSAS): Developed by the Japanese government;
    \item Korea Augmentation Satellite System (KASS): Currently in development by the Korea Aerospace Research Institute (KARI);
    \item Southern Positioning Augmentation Network (SPAN): Currently in development by Australia and New Zealand;
    \item ASECNA's SBAS for Africa and Indian Ocean (A-SBAS): Currently in development by the Agency for Air Navigation Safety in Africa and Madagascar (ASECNA), Nigerian Communications Satellite Ltd. (NIGCOMSAT) and Thales Alenia Space~\cite{a_sbas_2021}.
\end{itemize}
It should also be noted that the services provided by all these systems are compatible and interoperable, which means all the systems themselves also are~\cite{novatel_gnss,kaplan_2017,sbas_euspa_2021}.
% ------------------ fim da PARTE LARANJA.

% Ground Based Augmentation System: %meter wiki disto:
% Ground-Based Augmentation System (GBAS) aim at enhancing GNSS service levels for aviation during approach, landing and departure phases, as well as for surface operations. They have a local coverage (e.g. the surroundings of an airport) with the primary objective of meeting aviation requirements for the aforementioned operations and phases, in terms of accuracy, integrity and safety. Actually, whereas the main goal of GBAS is to provide integrity assurance, it also increases the accuracy with position errors below 1 m (1 sigma).

% The ground infrastructure includes two or more GNSS receivers which collect pseudoranges for all the primary GNSS satellites in view and computes and broadcasts differential corrections and integrity-related information for them based on its own surveyed position. These differential corrections are transmitted from the ground system via a Very High Frequency (VHF) Data Broadcast (VDB). The broadcast information includes pseudorange corrections, integrity parameters and various locally relevant data such as Final Approach Segment (FAS) data, referenced to the World Geodetic System (WGS-84). Any aircraft (within the area of coverage of the ground station) may use those corrections to compute their own measurements to compute a (differentially corrected) position, which at the same time is used to generate navigation guidance signals.

% GBAS and SBAS are both GNSS safety critical systems for civil aviation which share similar principles. The main difference comes from the fact that GBAS provides local corrections to the satellite pseudoranges using just ground infrastructure in the vicinity of the served airport, whilst SBAS broadcasts corrections to the different components of the pseudorange error valid for an area as big as a continent; the price to pay is that the SBAS infrastructure needs tens of sensor distributed in the augmented area and two or more GEO satellites to broadcast the information.


% %wiki: ..................
% A Satellite Based Augmentation System (SBAS) is a civil aviation safety-critical system that supports wide-area or regional augmentation - even continental scale - through the use of geostationary (GEO) satellites which broadcast the augmentation information[1][2]. A SBAS augments primary GNSS constellation(s) by providing GEO ranging, integrity and correction information. While the main goal of SBAS is to provide integrity assurance, it also increases the accuracy with position errors below 1 metre (1 sigma).

% The ground infrastructure includes the accurately-surveyed sensor stations which receive the data from the primary GNSS satellites and a Central Processing Facility (CPF) which computes integrity, corrections and GEO ranging data forming the SBAS signal-in-space (SIS). The SBAS GEO satellites relay the SIS to the SBAS users which determine their position and time information. For this, they use measurements and satellite positions both from the primary GNSS constellation(s) and the SBAS GEO satellites and apply the SBAS correction data and its integrity.

% The augmentation information provided by SBAS covers corrections and integrity for satellite position errors, satellite clock - time - errors and errors induced by the estimation of the delay of the signal while crossing the ionosphere. For the errors induced by the estimation of the delay caused by the troposphere and its integrity, the user applies a tropospheric delay model.

% SBAS systems, such as EGNOS or WAAS, are usable for the safety-critical task of guiding aircraft -vertically as well as horizontally- during different operations, including landing approaches (approach with vertical guidance, APV).
% %............................

% Used to provide integrity assurance;
% Used to increase accuracy and to reduce position errors to less than 1 meter.
% "Augmentation of a global navigation satellite system (GNSS) is a method of improving the navigation system's attributes, such as accuracy, reliability, and availability, through the integration of external information into the calculation process."

\subsection{Real-Time Kinematic (RTK)}\label{sec:II_gnssAug_rtk}

%parte AMARELA: --------------------
As mentioned in section~\ref{sec:II_gnssAug}, carrier-based methods are the go-to when dealing with navigation in a real-time scenario, which perfectly describes Real-Time Kinematic positioning, a method that is precisely one of the main topics of this Dissertation.

RTK fits, as indicated, in situations where high precision is indispensable for the performance of a certain task (e.g. use of a UAV to distribute pesticides in an agricultural field); as such, the accuracy levels obtained should be of the smallest order possible -- recalling Table~\ref{tab:GNSS_sys_errors}, which, from an RTK point of view, would be unacceptable.

%meter imagem da Figura 42 - base station + rover RTK
\begin{figure}[ht]
	\centering
	\includegraphics[width=1.0\textwidth]{Chapters/Figures/demo.png}
	\caption{RTK positioning.}
	\label{fig:rtk_workings}
\end{figure}

Looking at Figure~\ref{fig:rtk_workings}, the base station-rover pair can be seen again, so it is possible to draw an analogy to Figure~\ref{fig:dgnss_corrections}. The basic operation behind this precise positioning method consists, essentialy, in the process characterized by the detection and correction of positioning errors through the determination of ranges to GNSS satellites -- again based on equation (\ref{eq:2_1}). Conceptually, the calculation of these ranges is performed as explained in section~\ref{sec:II_gnssAug}, with regard to the carrier-based methodology: by calculating the number of cycles of the carrier wave between the satellite and the receiver installed on the rover ($n$) and multiplying it by its own wavelength ($\lambda$), where

\begin{equation}\label{eq:2_4}
    \lambda = \frac{v}{f}\medskip
\end{equation}
In this case, $v$ and $f$ are the propagation velocity and frequency of the carrier wave, respectively.

However, the calculations performed will inevitably continue to have range errors related to the already known effects intrinsic to satellite clocks and the Earth's atmosphere. It is precisely at this stage that the rover's positioning corrections by the base station come in, corrections that are then transmitted through, typically, a radiofrequency link.

As explained in~\ref{sec:II_gnssAug}, the use of a technique based on carrier waves often leads to ambiguity problems, so their resolution must be prioritized, so that it is possible to determine the number of complete cycles ($n$), properly taking advantage of the calculated corrections -- this can be done with the help of PRN codes, which, in high-precision GNSS receivers, happens almost instantly.

Thus, it is known that the ability to determine the exact position in RTK positioning depends on the correction of errors derived from the sources already mentioned, which confirms the use of the term ``differential'', as used in the DGNSS technique, since the determination of the rover's position in both methods depends on a base station. However, the crucial difference between these methods lies in the fact that the latter makes use of a code-based methodology, which, as is known, leads to results with lower precision than those obtained using a carrier-based methodology. Note also from Figure~\ref{fig:rtk_workings} that there is yet another factor to consider for the accuracy of measurements, which is the distance from the base station itself to the rover, known as ``baseline''. The variation of this distance does, as can be intuitively deduced, also vary the accuracy of the determined location. When compared to RTK positioning, the DGNSS positioning technique allows baselines, which is due to the use of the code-based methodology. However, the trade-off occurs, as we know, at lower precision values.

There is still another factor to consider for accuracy, which, in turn, is not represented in Figure~\ref{fig:rtk_workings}: the positioning of the base station. This parameter should not, in any way, be neglected, since, when connections are established through radiofrequency with a rover, the propagation of correction information will be done, as the term implies, through radio waves; due to their nature, these waves may suffer the same problems already observed in the propagation of waves from GNSS satellites -- such as interference and multipath --, so the location selection is important to minimize all negative effects. Another way to do this could also be to improve the quality of the rover's antennas.
% ------------------ fim da PARTE AMARELA.

%MUDAR, ESTÁ COPIADO!
So, the difference between RTK and DGPS is that DGPS is the traditional differential GPS.
RTK is a specific type of DGPS.
but it uses a newer technology than the traditional DGPS.
RTK stands for real-time kinematic and commonly uses the RTCM protocol.
The traditional DGPS uses an older antiquated protocol while RTK uses a newer algorithm, and the protocol is based on RTCM3. 
%____________

- Used to improve the accuracy of standalone GNSS receivers. Traditional GNSS receivers can only determine the position with na accuracy of about 2-4 meters (?). RTK provides centimeter accuracy.
- GNSS receiver measure how long it takes for a signal to travel from a satellite to the receiver. Due to the presence of atmosphere between the satellite and the receiver, the transmitted signals are slowed down and are introduced to perturbations. With this in mind, one can immediately assume that transmission times will differ according to the weather at the time of the event. That is the reason why a standalone receiver has a hard time determining its position accurately. RTK is a Technology that solves this issue.
- 2 receivers are used in RTK. One of them is stationary, the other is a moving rover.
- Real Time Kinematic (RTK) is a GPS correction technology technique that provides real-time corrections to location data as the drone is surveying and capturing images from a site.    


\subsubsection{Networked RTK}\label{sec:II_ntrip}

Obtaining position corrections with RTK can also be done in a way that does not require the aid of a base station set up by the end-user -- through a process known as Network RTK.
Network RTK is a technique that relies on the use of stations permanently fixed at known and spaced positions -- which are, essentially, base stations -- to obtain positioning corrections over the internet. Thus, this process can be seen for what it literally is: a way to obtain positioning corrections through a network of virtual base stations, which immediately leads to the conclusion that long baselines cease to be a problem.

As stated in section~\ref{sec:II_gnss_comm}, if a receiver relies solely on satellite signals for positioning, the rate at which information will be received will be rather low. Relying on internet connection to get position corrections is a great way to bypass such problem. The use of a CORS (Continuously Operating Reference Station) network is an example of networked RTK. Its operation is based on the use of stations that are permanently installed in previously known and properly inter-spaced locations, making regular observations of the GNSS satellites correcting any errors based on such.
Most countries have systems of this kind, which are openly accessible to the public. In addition to helping with positioning corrections, CORS stations can also transmit (through knowledge of their position) information about changes in the Earth's surface, thus helping to detect any tectonic movements -- which, in consequence, assists in measuring earthquakes and volcanic eruptions. An example of such is the portuguese CORS network (ReNEP\footnote{Accessible through \url{https://renep.dgterritorio.gov.pt}.}); consisting of 47 permanent GNSS stations throughout the country, this system provides not only navigation information byt also information about the tectonic plate on which a particular station is located~\cite{ReNEP_ppt_2018}.

%daniel:
The next step when dealing with the virtual base stations is for the rover to effectively fetch their positioning data from the internet so that the corrections can be made. For that, another type of network can be used, which can be seen in Figure~\ref{fig:ntrip_network}: an NTRIP network (Network Transport of RTCM via Internet Protocol). NTRIP networks are used to effectively make the use of a second GNSS receiver obsolete. This is possible due to the fact that this type of network is capable of establishing a virtual link between the rover and the base stations through the internet. The former then detects and corrects the errors, subsequently forwarding the treated data to the rover. The connection can be made through an already familiar radiofrequency link -- this time, to the internet itself -- or another chosen method.

Figure~\ref{fig:ntrip_network} also represents the three main sectors that constitute an NTRIP-based environment:

\begin{itemize}
    \item Base: Usually a CORS reference/base station -- assured to be in continuous operation; used to detect errors and calculate precise solutions;
    \item NTRIP caster: An HTTP internet service that forwards the connection data through the internet;
    \item Rover: The client's receiver that expects real-time precise positioning solutions.
\end{itemize}
RTCM is a communications protocol~\cite{rtcm_2003}

%meter imagem NTRIP da Emlid:
\begin{figure}[ht]
	\centering
	\includegraphics[width=1.0\textwidth]{Chapters/Figures/demo.png}
	\caption{Network RTK positioning using NTRIP.}
	\label{fig:ntrip_network}
\end{figure}

ntrip client, caster, ?
%emlid:
The Networked Transport of RTCM via Internet Protocol or NTRIP network was developed by the German Federal Agency for Cartography and Geodesy in 2004. NTRIP allows your rover to accept corrections over the Internet with no need for the second local receiver acting as a base.

NTRIP includes three main components of the system: base, caster, and client's rover. Usually, in this case, there's a stationary continuously operating reference station or CORS. The data is then sent to the NTRIP caster, where it is retransmitted through the Internet port to the client rover connected via a particular port and authorized. 

% https://www.youtube.com/watch?v=uytd48Vb-fs&ab_channel=RamiTamimi
"NTRIP (Network Transport of RTCM via Internet Protocol) and CORS (Continuously Operating Reference Station) are forms of RTK differential correction that are done using a cellular modem and base station network."
"how to obtain high accuracy positioning utilising just one GNSS recevier?
As described in section (\textbf{ref}) , it is possible obtain high accuracy positioning by setting up two GNSS receivers, which will act as a base station and a rover. Therefore, this method requires double the cost of the method presented in this section. There is a way to use only one GNSS receiver and have RTK-enabled positioning, through the use of a CORS (Continuously Operating Reference System) network.% corte aqui
These stations are permanentely set up on a single known location, and continue to observe satellites and perform corrections based on any errors that they observe. Most municipalities, states and countries own these systems and allow public access to anyone that sets up na account.

% next:
Depending on the implementation, positioning data from the permanent stations is regularly communicated to a central processing station.
On demand from RTK user terminals, which transmit their approximate location to the central station, the central station calculates and transmits correction information or corrected position to the RTK user terminal.
The benefit of this approach is an overall reduction in the number of RTK base stations required. Depending on the implementation, data may be transmitted over cellular radio links or other wireless medium.



% falar de todos os topicos que estejam no doc Word NTRIP
The fact that beRTK is connected to the NTRIP network allows you to start with a more precise idea of its location and therefore faster convergence.
"Real Time Kinematic technique requires 2 receivers. One of them is stationary and is called "base station", the other one is "rover". The base station measures errors, and knowing that it is stationary transmits corrections to the rover (refer to How RTK works for more information about RTK). Sometimes CORS and NTRIP networks take the place of traditional base stations. They provide accurate absolute position and send corrections over the Internet. Typically the distance between the reference station and local rover shouldn't exceed 10-15 km due to the ionospheric effect. So if the reference station is located too far or simply is absent in the area you will need a local base station. Other advantages of your own base are independence from the Internet connection and lack of NTRIP subscription fees."

- If the accurate absolute position of the base has been determined only after the job has been done, the offset of the map can be determined and corrected.


If there are no NTRIP stations within a radius of (?) from the intended mission site, a base station will be needed in order to obtain the precise positioning through RTK, as described in~\ref{sec:II_gnssAug_rtk}.
"it is very fast to obtain a fix. Rather than utilising our own base station for the corrections, the public NTRIP network is used, and the receiver (rover) will be connected to a base station in that network."
Fazer uma imagem parecida a esta: % ver Word

An internet based application that makes the RTCM Correction data from the CORS stations available to anyone with an internet connection and the appropriate log on credentials to the NTRIP server. Typically uses a mobile link to get to the internet and the NTRIP server. % https://www.teejet.com/CMSImages/TEEJET/documents/technical-updates/98-01410%20r0%20en%20tech%20update%20ntrip%20rx610.pdf

% book kaplan_2017:
As mobile packet-switched cellular networks proliferated around the world,
delivery of DGNSS and PPP data through the Internet Protocol (IP) has become increasingly popular. Section 12.5.3 describes one last widely used RTCM standard,
which is for the Networked Transport of RTCM via IP (NTRIP).

%se calhar nao sao precisas estas subsubsections! tirar!:

%references: \cite{,ntrip_agleader}

\subsubsection{Frequency Bands}\label{sec:II_gnssAug_rtk_freqbands}
- L1, L2, ... bands

\subsubsection{Initialisation Time}\label{sec:II_gnssAug_rtk_inittime}
- RTK Initialisation Time

\subsubsection{Single-band vs Multi-band}\label{sec:II_gnssAug_rtk_smband}
- Single-band vs Multi-band receiver

\subsubsection{Baseline}\label{sec:II_gnssAug_rtk_baseline}
Baseline in RTK mode and Baseline in PPK mode -- for different projects, a different distance from the rover to the base might be needed. Working near a city is more likely to have base station stations nearby. However, when working in rural areas, base stations are likely to be further away.
Multi-band receivers can work at a longer baseline due to the use of multiple satellite constellations -- as these help in the correction of the readings taken by the base, as mentioned before (earlier?). beRTK can operate with the baseline up to 2.5 km.
fazer uma imagem parecida a esta:
% meter imagem da baseline Emlid

\subsubsection{Accuracy}\label{sec:II_gnssAug_rtk_accuracy}
H: 7 mm + 1 ppm
V: 14 mm + 1 ppm	MEANING OF THIS???
% http://www.apegm.mb.ca/pdf/PD_Papers/GNSSPositioning.pdf
Both single-band and multi-band receivers are capable of centimeter-level absolute accuracy. The main difference is that more factors can influence the stable fix solution in the single-band receiver. Thus, when using a single-band receiver, you can obtain the same absolute accuracy, but only if you have reasonable working conditions.

\subsubsection{PPM}\label{sec:II_gnssAug_rtk_ppm}
PPM expresses a standardized measurement of error -- in millimeters per 1,000 meters -- in relation to orthometric heights. For instance, an orthometric height that has a 2 PPM error rate would indicate an error in measurement equal to 2 millimeters per 1,000 meters traveled. So, if a mountain resort located 1,000 meters inland had a PPM of 2 millimeters, the orthometric height, or elevation, indicated would be accurate to within 2 millimeters.
% https://sciencing.com/difference-between-agl-msl-8524698.html
% https://unstats.un.org/unsd/geoinfo/ungegn/docs/_data_icacourses/_HtmlModules/_Selfstudy/S06/S06_03a.html

\subsubsection{LoRa -- meter na footnote}\label{sec:II_gnssAug_rtk_LoRa}
``There are number of communication technologies available for interaction between IoT devices today, and the most popular ones are Wi-Fi and Bluetooth. But the problem with Wi-Fi and Bluetooth technology is high power consumption. They also have other limitations like limited range, limited access points etc. ESP8266 module is the most popular Wi-Fi module used in IoT devices, using which we have previously built lot of IoT projects.

Cellular networks also have the same problems of high power consumption and both LAN and Cellular network are quite expensive to cover a wide area. The IoT industries introduced lots of technologies, but none of them was ideal for IoT devices, as they needed to transmit information to long distance without using much power, until the LoRa technology was introduced. LoRa Technology can perform very-long range transmission with low power consumption.

LoRa (Long Range) is a wireless technology that offers long-range, low power, and secure data transmission for M2M (Machine to Machine) and IoT applications. LoRa is a spread spectrum modulation technology that is derived from chirp spread spectrum (CSS) technology. LoRa can be used to connect sensors, gateways, machines, devices, etc. wirelessly. In Europe region, it operates in the 868 MHz band.'' % https://iotdesignpro.com/projects/lora-communication-between-two-arduino-using-LoRa-Module-SX1278

\subsection{Post-Processed Kinematics (PPK)}\label{sec:II_gnssAug_ppk}
PPK is na alternative technique to RTK. With PPK workflow, accurate positioning does not happen in real time, since all algorithms are applied afterwards. Both base station on the ground and rover (usually an UAV) record raw GNSS logs, which are then processed to receive na accurate positioning track.
%meter/fazer imagem do Word:

- PPK is mainly used for UAV mapping.
- Offers a more flexible workflow, allowing to run the processing multiple times using different settings the processing is applied on the logs returned by the both the base station and the rover used on (the?) field.
- PPK allows the inspection of much wider areas, which is why the baseline is greater than the baseline available while working in RTK mode.
- Post Processed Kinematic (PPK) is a GPS correction technology technique that corrects location data after it is collected and uploaded. The data can be uploaded to the cloud for processing or processed using specialise software on your desktop after the flight has been concluded.

\subsection{Precise Point Positioning (PPP)}\label{sec:II_gnssAug_ppp}
- A standalone receiver finds out its position relying on the data obtained from satellites only. Along with raw data from those satellites, the receiver gets navigation messages with satellite clock offset, ionospheric and tropospheric corrections (atmospheric-related disturbances), etc. Due to information about these offsets, the receiver may calculate its position with a few meters' accuracy. If there were (was) no navigation data, the accuracy would be much worse.
- In RTK and PPK, these offsets might be eliminated since both (the) base station and the rover operate in quite similar conditions.
- PPP allows the single receiver (rover) to achieve high-level accuracy without the use of corrections from the base station.
- To calculate the coordinates, PPP uses the same data that is provided by the navigation message but much more accurate. Thereby, the single receiver (rover) might determine its position with a centimetre-level accuracy using only raw data and precise ephemerides and clock offsets provided by a PPP service.
- The PPP technique is commonly used for determining the absolute base position for further RTK/PPK surveys.

%



% \section{Battery System}\label{sec:II_battery}
% % falar de todos os topicos que estejam no doc Word Battery

% The formula is (Wh)/(h) = (W). For example, if you have 100 Wh for a duration of 2 hours, then the wattage is (100)/(2) = (50) Watts.
% (Watthours is a measure of energy and watts is a unit of power. Power multiplied by time is enery).

% como cada uma das baterias atualmente em uso is rated for (as?) 7.4V, 1070 mAh, that corresponds to 7.918 Wh.

% % LiFePO$_4$ better than Li-ion batteries?
% ``The LiFePO$_4$ battery has the edge over lithium-ion, both in terms of cycle life (it lasts 4-5x longer), and safety. This is a key advantage because lithium ion batteries can overheat and even catch fire, while LiFePO4 does not''% citation needed

% \subsection{USB Type-C}\label{sec:II_usb_c}
% % ler wiki do USB-C e derivar os topicos daí
% % depois ir ao IEEEXplore procurar papers que dêem backup

% \subsubsection{Power Delivery}\label{sec:II_usb_c_PD}
% % PD is a protocol

\section{Current Solutions}\label{sec:II_curr_solutions}
It is necessary to adapt all the solutions abordadas to the current software component; this means that all the solutions have to be implemented.

16. explicar o que cada célula do excel significa, tanto as que considerei mais importantes como as outras

as stated in section~\ref{sec:II_gnssAug_ppk}, Post-processing generally results in a more accurate, comprehensive solution than is possible in real-time, as it is possible to observe through Table \textbf{TABLE?}.
% gravar como pdf/imagem; meter no corpo do texto
% na versao final a tabela sera feita completamente em latex
