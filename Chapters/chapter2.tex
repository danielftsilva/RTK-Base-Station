%!TEX root = ../template.tex
%%%%%%%%%%%%%%%%%%%%%%%%%%%%%%%%%%%%%%%%%%%%%%%%%%%%%%%%%%%%%%%%%%%%
%% chapter2.tex
%% NOVA thesis document file
%%
%% Chapter with the template manual
%%%%%%%%%%%%%%%%%%%%%%%%%%%%%%%%%%%%%%%%%%%%%%%%%%%%%%%%%%%%%%%%%%%%

\typeout{NT FILE chapter2.tex}

% \printbibliography[heading=subbibliography, segment=\therefsegment, title={\bibname\ for chapter~\thechapter}]
\glsresetall
\chapter{State of the Art}\label{cha:II_SotA}
% escrever aqui qlq coisa de introduçao:

\section{Global Navigation Satellite System}\label{II_gnss}

%MUDAR, ESTÁ COPIADO!
So, the difference between RTK and DGPS is that DGPS is the traditional differential GPS.
RTK is a specific type of DGPS.
but it uses a newer technology than the traditional DGPS.
RTK stands for real-time kinematic and commonly uses the RTCM protocol.
The traditional DGPS uses an older antiquated protocol while RTK uses a newer algorithm, and the protocol is based on RTCM3. 
%____________

Nowadays, whenever someone wishes to know their current location on Earth, just a few, effortless taps on a smartphone will be enough; this is often associated with GPS, which has been around for many years. In a more general manner, this technology can be described as a Global Navigation Satellite System, or GNSS. This term refers to any satellite constellation that can be used to help navigate throughout the world (as the name suggests). derives from a set of artificial satellites that comprises a network, and such network belongs 

to the average smartphone nowadays works as a GNSS receiver.
global navigation satellite system comprises a network of satellites that continuously orbit the Earth, constantly emmiting radio-frequecy (hyphen?) signals carrying information about their current status, position in space and precise time.
This information is acheived through atomic clocks, installed within the satellite itself.

smartphone = single-band ou multi-band?
how many constellations;
how many satellites in each constellation;

\section{RTK}\label{II_rtk}

\section{NTRIP}\label{sec:II_ntrip}

\section{Battery System}\label{sec:II_battery}

\section{Current Solutions}\label{sec:II_curr_solutions}

% \begin{tabularx}

% \end{tabularx}

\begin{table}[ht]       % provavelmente devo meter as palavras "parameter"/"base station"?
	\centering
    \captionsetup{justification=centering}
    \caption{Compilation of the most relevant solutions available in the market (as of the date of the present document).}
	\label{tab:current_solutions}
	\begin{tabular}{cccccc}
		\toprule
		\multicolumn{1}{c}{\textbf{beRTK (Beyond Vision)}} & \textbf{Reach RS2 (Emlid)} & \textbf{D-RTK 2 (DJI)} & \textbf{HiPer V (Topcon)} & \textbf{S990A GNSS Receiver (Stonex)}\\
		\midrule
		\textbf{Exterior} & 2 & 2 & 1 & 1 & 1 \\
		\textbf{Correction Technology} & 2 & 2 & 1 & 1 & 1 \\
        \textbf{Supported Constellations} & 2 & 2 & 1 & 1 & 1 \\
        \textbf{Positioning Accuracy} & 2 & 2 & 1 & 1 & 1 \\
        \textbf{Range of Correction Transmission} & 2 & 2 & 1 & 1 & 1 \\
        \textbf{Communication} & 2 & 2 & 1 & 1 & 1 \\
        \textbf{Data} & 2 & 2 & 1 & 1 & 1 \\
        \textbf{Power Supply, EPI,battery)} & 2 & 2 & 1 & 1 & 1 \\
        \textbf{Autonomy} & 2 & 2 & 1 & 1 & 1 \\
        \textbf{Dimensions, Weight} & 2 & 2 & 1 & 1 & 1 \\
        \textbf{Environmental} & 2 & 2 & 1 & 1 & 1 \\
		\bottomrule
	\end{tabular}
\end{table}
