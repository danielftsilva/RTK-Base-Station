%!TEX root = ../template.tex
%%%%%%%%%%%%%%%%%%%%%%%%%%%%%%%%%%%%%%%%%%%%%%%%%%%%%%%%%%%%%%%%%%%%
%% chapter5_ConclusionsFuture.tex
%% NOVA thesis document file
%%
%% Chapter with the Conclusions and Future Work part
%%%%%%%%%%%%%%%%%%%%%%%%%%%%%%%%%%%%%%%%%%%%%%%%%%%%%%%%%%%%%%%%%%%%

\typeout{NT FILE chapter5_ConclusionsFuture.tex}

\chapter{Conclusions and Future Work}\label{cha:chapter5_ConclusionsFuture}

Considering the impact of robotic devices in the most diverse types of fields (whether scientific or not), their precise development is of great importance, so the definition and implementation of a project in an orderly and careful way is quite valuable.

Agricultural work, surveillance, mapping, etc. performed using robotic devices always brings many advantages, taking into account that the yield obtained turns out to be equal or (often) superior to that of a human being.
Thus, in order to develop a device capable of doing so, the first priority will be the study of all the relevant information available, as well as all the technology known on the subject -- the study of the most relevant GNSS positioning technologies covered in Chapter~\ref{cha:chapter2_SotA} covers this aspect, corresponding to the state of the art.
Having done this, the next step is related to the establishment of requirements to be fulfilled for the system to be developed, so that the solution obtained fulfills all the purposes initially intended -- Section~\ref{sec:III_requirements}. To this end, assimilating a chronological order made up of phases allows an estimate of the interdependencies of each one of them, as well as defining expected deadlines (Tables~\ref{tab:workflow} and~\ref{tab:gantt}).\\

\par Successfully accomplishing complicated, hard, and time-consuming tasks is, without a doubt, something that anyone can be proud of. Nonetheless, envisioning and developing a solution capable of performing them as efficiently as possible is undoubtedly the approach that characterizes an engineer. In addition to providing valuable insights into the known literature on navigation through GNSS services -- most emphatically precise positioning --, this document also presents an estimate of the approach to be taken for the development of the project, as well as the chronological order that is considered the wisest for such.

%           PODE SER USADO NA CONCLUSAO: about RTK
% "Real Time Kinematic technique requires 2 receivers. One of them is stationary and is called "base station", the other one is "rover". The base station measures errors, and knowing that it is stationary transmits corrections to the rover (refer to How RTK works for more information about RTK). Sometimes CORS and NTRIP networks take the place of traditional base stations. They provide accurate absolute position and send corrections over the Internet. Typically the distance between the reference station and local rover shouldn't exceed 10-15 km due to the ionospheric effect. So if the reference station is located too far or simply is absent in the area you will need a local base station. Other advantages of your own base are independence from the Internet connection and lack of NTRIP subscription fees."
% - If the accurate absolute position of the base has been determined only after the job has been done, the offset of the map can be determined and corrected.

1. Calcular para uma charge current de 1020mA e nao de 4000mA -- fazer as contas no excel.
	
2. This results in a value of R15 of approximately 5.36k$\Omega$, however, at the time of soldering, the closest standard resistor value available was 6.2k$\Omega$, which, when applying the same previous law to a variation of the equivalent circuit of Figure~\ref{fig:voltage_divider_1} results in a voltage between R10 and R15 of approximately 1.15V -- sufficient to drive Q3.

3. Meter condensadores de desacoplamento --> como o modulo 3.3V funciona por pulsos (btw tudo no esquematico é digital), vai existir ruido consideravel quando a tensao for puxada/convertida --> os condensadores de desacoplamento servem para atenuar esse ruido.
