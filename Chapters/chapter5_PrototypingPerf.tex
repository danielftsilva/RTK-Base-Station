%!TEX root = ../template.tex
%%%%%%%%%%%%%%%%%%%%%%%%%%%%%%%%%%%%%%%%%%%%%%%%%%%%%%%%%%%%%%%%%%%%
%% chapter5_PrototypingPerf.tex
%% NOVA thesis document file
%%
%% Chapter with the Prototyping and Performance Evaluation Results part
%%%%%%%%%%%%%%%%%%%%%%%%%%%%%%%%%%%%%%%%%%%%%%%%%%%%%%%%%%%%%%%%%%%%

\typeout{NT FILE chapter5_PrototypingPerf.tex}

\chapter{Prototyping and Performance Evaluation Results}\label{cha:chapter5_PrototypingPerf}

%So neste capitulo e que se deve descrever a operacao do circuito.
Upon completion of the circuit design phase, the next step starts to dictate the physical shape of the prototype.

\section{PCB Layout}\label{sec:51_PCBlayout} % se calhar tiro "Design"
%provavelmente este capitulo deve ficar no final do Chapter 4, porque layout também é hardware design

\subsection{Footprint Assignment and Placement}\label{sec:511_Placement}

Once every component symbol in the schematics is correctly referenced and the electrical rules check reports no errors nor warnings, KiCad provides a footprint assignment tool, to attribute the desired footprints to the project's parts. Since the new beRTK\textsuperscript{\textregistered} circuit is based around fourteen main ICs and modules, selecting the correct footprints for each of these was the starting point for the layout phase. For that, datasheets for ICs and modules usually feature a section dedicated to presenting the recommended footprint for their component. This facilitates designers' work, as KiCad already provides a vast list of footprint libraries with many footprints to choose from, and therefore the attribution boiled down to a simple matter of finding the correct matches, referring to the datasheets. Figure~\ref{fig:footprint_AP64501} shows the recommended solder pad pitch and dimensions for the AP64501 buck converter addressed in Section~\ref{sec:3214_AP64501}, an example of what is taken as reference when assigning footprints.

% meter aqui footprint LTC4012 - datasheet
\begin{figure}[h]
	\centering
	\includegraphics[width=0.7\textwidth]{Chapters/Figures/chapter5/footprint_AP64501.pdf}
	\caption{Recommended solder pad pitch and dimensions for the AP64501 synchronous buck converter~\cite{AP64501}.}
	\label{fig:footprint_AP64501}
\end{figure}

For the ICs and modules whose footprints are not provided by default by the software, internet research was conducted in order to find them -- \url{https://www.snapeda.com/} and \url{https://pt.mouser.com/} were the two websites accessed for such purpose, since these are known to provide trustworthy footprints.

Regarding resistors, unless otherwise stated, all should have a 0805 \gls{SMD} package footprint and a tolerance of 5\%. The standard 0805 package size measures $0.08 \times 0.05$ inches (length $\times$ width), which corresponds, in the metric system, to a 2012 package size, at $2.0 \times 1.25$ millimetres (length $\times$ width).
Capacitors should also have a 0805 SMD package footprint and be rated for 50V, unless otherwise stated.
As for the remaining components, i.e. MOSFETs, diodes, inductors, ferrites, crystal and connectors, their footprint depends on the application.
Datasheets of ICs and modules usually refer the characteristics and/or models to use in a typical implementation.

To effectively design the board's layout, KiCad provides a PCB editor that places the assigned footprints in single editing window. This premature placement must then be rearranged according to the respective component's location in the circuit, in the most efficient way possible among other partner components. The PCB editor also displays white lines (known as ``ratsnest lines'') that connect the different components' pads to each other, just as defined in the schematic design. This subsequently helps in the routing process (i.e. effectively connecting every component with tracks (also known as ``traces''), vias and zones). Figure~\ref{fig:ratsnest} shows an early version of the layout for the external power supply region of the system, which is connected to the LTC4012 and to the external power voltage reference (top left area of the schematic of Figure~\ref{fig:LTC4012_circuit}).

\begin{figure}[h]
	\centering
	\includegraphics[width=0.8\textwidth]{Chapters/Figures/chapter5/ratsnest.png}
	\caption{Unconnected footprints for the external power supply region of the system (early version).}
	\label{fig:ratsnest}
\end{figure}


%s--Ss--Ss--Ss--Ss--Ss--Ss--Ss--Ss--Ss--Ss--Ss--Ss--Ss--Ss--Ss--Ss--Ss--Ss--S
\subsubsection{Control Unit and USB 2.0 Hub}\label{sec:5111_CM4_LAN9514}

%0. falar do placement estar concluido:
To arrange all components in the best way possible, one has to take into account a provisional routing, to achieve a good placement in the least amount of iterations possible. For that, it's considered good practice to start by imagining and experimenting with placing and routing of the high-speed zones of the circuit first. For this project, that would concern the CM4's high-speed side and the LAN9514's USB side. Figure~\ref{fig:placement_CM4_LAN9514} focuses on the final component placement of the CM4 module and the LAN9514 hub on the board.

\begin{figure}[h]
	\centering
	\includegraphics[width=0.7\textwidth]{Chapters/Figures/chapter5/placement_CM4_LAN9514.png}
	\caption{Final placement of the CM4 and LAN9514 circuits' components.}
	\label{fig:placement_CM4_LAN9514}
\end{figure}%placement_CM4_LAN9514

The location of the peripheral ports (HDMI and USB) must also be accounted for. It is common for these to be placed near the limits of the board so that user access is facilitated.

After placing the downstream USB connectors near the right edge of the board, the LAN9514 hub was placed to the left of these, in way that it would be more-or-less equidistant from each to account for the routing of the high-speed USB data traces. After defining a position for the hub, its most important component was placed: the crystal oscillator XT1. This part must be the closest possible to its respective hub's pins, in order to settle a trustworthy clock frequency from which the IC will rely on. Followed by the crystal's capacitors, the remaining components were placed the nearest possible to the hub.

Setting the CM4 immediately to the left of the USB hub, choosing to locate the HDMI port on the left edge of the board, near the CM4's HDMI pins, was straightforward.

The power and activity LEDs (D15 and D14, respectively) were placed near the top left edge of the board (i.e. top left of Figure~\ref{fig:placement_CM4_LAN9514}), a simple and easy-to-remind location, specially for the testing phase.

All the CM4's test points (e.g. +3V3, WL\_nDisable, GPIOx, etc.) are located either near the top or bottom of the module (MOD1) -- also easy-to-remind and to reach locations.

Regarding the LAN9514's bypass capacitors, these were placed near the top-middle of the board, as well as above and below the downstream USB connectors (also visible in Figure~\ref{fig:placement_CM4_LAN9514}).


%s--Ss--Ss--Ss--Ss--Ss--Ss--Ss--Ss--Ss--Ss--Ss--Ss--Ss--Ss--Ss--Ss--Ss--Ss--S
\subsubsection{Power Selector and Battery Balancer}\label{sec:5112_LTC4012_BQ29209}

The next step taken was the placement of the power selector circuit. In the applications information sections of the LTC4012 datahseet (\cite{LTC4012}), a list of PCB layout considerations are provided. Regarding these, it is recommended to follow a specific priority order, as to ensure a proper layout. The notes go through the best possible layout for the switch node of the battery charger and its importance in the design. This switch node corresponds to the point where diode D4 and inductor L1 meet (see Figure~\ref{fig:LTC4012_circuit}), and is directly connected to pin SW of the LTC4012. The suggested priority list for the LTC4012 layout starts with notes on the correct placement of the switching FETs (Q2 and Q9) relative to the IC and its input capacitors, running through the inductor, current sense resistors, output capacitors and notes regarding PCB layer, trace and via topology. Following these guidelines resulted in the component placement represented on the right side of Figure~\ref{fig:placement_Power_Selector_and_BQ29209}, where the switching FETs, input capacitors and inductor are clearly visible close and around the LTC4012 footprint.

%placement_Power_Selector_and_BQ29209
\begin{figure}[h]
	\centering
	\includegraphics[width=0.7\textwidth]{Chapters/Figures/chapter5/placement_Power_Selector_and_BQ29209.png}
	\caption{Final placement of the Power Selector and Battery Balancer circuits' components.}
	\label{fig:placement_Power_Selector_and_BQ29209}
\end{figure}

BQ29209's datasheet (\cite{bq29209}) also provides a dedicated section highlighting recommended layout suggestions for users. Among other notes, these encourage the designer to place the input capacitors as close as possible to the IC in order to filter out the most amount of noise possible. This resulted in the placement shown on the lower left-hand side of Figure~\ref{fig:placement_Power_Selector_and_BQ29209}.


%s--Ss--Ss--Ss--Ss--Ss--Ss--Ss--Ss--Ss--Ss--Ss--Ss--Ss--Ss--Ss--Ss--Ss--Ss--S
\subsubsection{Voltage Converter}\label{sec:5114_VoltageConverter}

Similar to the previous IC's datasheets,~\cite{AP64501} singles important guidelines for a thorough layout of the AP64501's circuit. These guidelines are laid out in list form, starting by addressing the recommended copper thickness for the carrier board for the IC (due to the maximum 5A output current), followed by the importance of the placement of the input capacitors and inductor (L2) as close as possible to their respective terminals on the converter. More specifically, the input capacitors should be placed close to the power supply (VIN) of the IC, and the inductor close to the switching terminal (SW) -- similar to the layout topology addressed by the LAN9514's datasheet. After this, the list focuses on output capacitors, feedback components, and finally PCB layers and vias. A layout schematic example is also provided, and is represented in Figure~\ref{fig:AP64501_layout_Datasheet}.

% meter aqui AP64501 layout - datasheet
\begin{figure}[h]
	\centering
	\includegraphics[width=0.5\textwidth]{Chapters/Figures/chapter5/AP64501_layout_Datasheet.pdf}
	\caption{Recommended layout for the AP64501 synchronous buck converter's circuit~\cite{AP64501}.}
	\label{fig:AP64501_layout_Datasheet}
\end{figure}

Taking all these notes into account resulted in the component placement depicted on Figure~\ref{fig:placement_VoltageConverter}.

%placement_VoltageConverter
\begin{figure}[h]
	\centering
	\includegraphics[width=0.6\textwidth]{Chapters/Figures/chapter5/placement_VoltageConverter.png}
	\caption{Final placement of the Voltage Converter circuit's components.}
	\label{fig:placement_VoltageConverter}
\end{figure}


%s--Ss--Ss--Ss--Ss--Ss--Ss--Ss--Ss--Ss--Ss--Ss--Ss--Ss--Ss--Ss--Ss--Ss--Ss--S
\subsubsection{Power Switch}\label{sec:5115_PowerSwitch}

The power switch circuit placement revealed itself to be very simple.
\cite{SN74LVC2G74DCTR} highlights that the unused inputs for the D-type flip-flop must be tied either to GND or VCC. Figure~\ref{fig:placement_PowerSwitch} shows the arrangement made for this circuit, which essentially surrounds the main ICs (the D-type flip-flop, the inverter, and the voltage regulator) with their function-defining components.

%placement_PowerSwitch
\begin{figure}[h]
	\centering
	\includegraphics[width=0.4\textwidth]{Chapters/Figures/chapter5/placement_PowerSwitch.png}
	\caption{Final placement of the Power Switch circuit's components.}
	\label{fig:placement_PowerSwitch}
\end{figure}


%s--Ss--Ss--Ss--Ss--Ss--Ss--Ss--Ss--Ss--Ss--Ss--Ss--Ss--Ss--Ss--Ss--Ss--Ss--S
\subsubsection{Human-Machine Interface}\label{sec:5116_HMI}

Figure~\ref{fig:placement_HMI} shows the placement of the LM2901 comparator's circuit (see left-hand side of Figure~\ref{fig:HMI_circuit}).
Placement for this part of the system also proved to be straightforward, due to the reduced number of components (mainly resistors), which were simply laid out around comparator U6.

As explained previously, the status voltage from the comparator flows to the HMI itself through the connection of two 2.54mm pitch pin headers, F4 and F5. F4 is placed close to U6, as well as to the top right corner of the board, on its opposite side (see Figure~\ref{fig:placement_HMI}, left). It is supposed to be connected to F5 to power all status-indication LEDs and push-button SW1, as well as to establish nets PWR\_SWITCH, PWR\_LED and EXT\_PWR. The right side of Figure~\ref{fig:placement_HMI} shows a small separate PCB just for the HMI. The reason for this separate board is due to its specific user-accessible location on the outside of the base station's casing.

%placement_HMI
\begin{figure}[h]
	\centering
	\includegraphics[width=0.8\textwidth]{Chapters/Figures/chapter5/placement_HMI.png}
	\caption{Final placement of the status voltage comparator (left) and HMI (right) circuits' components.}
	\label{fig:placement_HMI}
\end{figure}


%s--Ss--Ss--Ss--Ss--Ss--Ss--Ss--Ss--Ss--Ss--Ss--Ss--Ss--Ss--Ss--Ss--Ss--Ss--S
\subsubsection{GNSS module and Wi-Fi Transceiver}\label{sec:5117_ZED_XBEE}

Aside from connector F4, the GNSS module was the only system part to be placed on the back side of the main board, since this was the most convenient location due to its mechanical structure (i.e. because the module itself consists on a small PCB with a pin header soldered on it). The remaining components connected to the GNSS module's circuit, namely the RTK ISP connector, input capacitors C9 and C10, and diodes D6 and D7 are placed on the front side of the main board, near the opposite location of the module itself, along with the RTK data test points. All this can be seen on the top portion of the left side of Figure~\ref{fig:placement_ZED_XBEE}.

%placement_ZED_XBEE
\begin{figure}[h]
	\centering
	\includegraphics[width=0.8\textwidth]{Chapters/Figures/chapter5/placement_ZED_XBEE.png}
	\caption{Final placement of the GNSS module and Wi-Fi Transceiver circuits' components (left); Antenna keepout area for the XBee 3 RF module (right).}
	\label{fig:placement_ZED_XBEE}
\end{figure}

As for the Wi-Fi transceiver, i.e. the XBee 3 RF module,~\cite{XBee} provides layout design notes to help achieve the best antenna performance possible. Since the XBee module used in this project is known as the its ``micro chip'' version, the notes for this version are the ones to pay attention to. These state that non-metal enclosures are preferred, and that all metal parts of the system -- either internal or external in respect to the main board -- must be kept at the maximum distance possible from the module's antenna. For that,~\cite{XBee} also provides a graphic visualization of a recommended antenna keepout area to implement. In this area, only a copper pour and minimal routing (to connect the needed module's pins on that zone) are permitted. The right side of Figure~\ref{fig:placement_ZED_XBEE} shows such area (represented by a large rectangle).

After assessing these guidelines, the placement of this circuit's components was standard when compared to the previous sections, with output capacitors C11-C14 of the LDO U2 placed as close as possible to its output, as well as to the VCC input (pin 2) of the XBee module. All test points for the latter (DIO0-DIO3) are placed above it.

%   dizer que é importante o cristal estar mais perto possivel do LAN, porque o routing desta parte do circuito é critica;
%   importante as USB downstream ports estarem o mais perto possivel do LAN
%   ver se me lembro de mais regras...
%   definiu-se um limite experimental para a placa ($100 \times 70$ mm (length $\times$ width)) e acabou por se aumentar para $129 \times 90$ mm (length $\times$ width).

Concluding the placement of every circuit, the remaining placement-related task was to assign these to roughly specific locations on the board. Choosing the board's dimensions settles the placement phase. For the main board, the final size was set to $129 \times 90$ mm (length $\times$ width). As for the HMI board, the same $30 \times 60$ mm (length $\times$ width) measurements from the previous beRTK\textsuperscript{\textregistered} version were kept unaltered. Figure~\ref{fig:placement_FULL} shows the final placement of the entire system within the board's defined limits.

%placement_FULL
\begin{figure}[h]
	\centering
	\includegraphics[width=0.6\textwidth]{Chapters/Figures/chapter5/placement_FULL.png}
	\caption{Final placement of the new beRTK\textsuperscript{\textregistered} system's circuit, within the board's defined limits.}
	\label{fig:placement_FULL}
\end{figure}


%sSsSsSsSsSsSsSsSsSsSsSsSsSsSsSsSsSsSsSsSsSsSsSsSsSsSsSsSsSsSsSsSsSsS
\subsection{Routing}\label{sec:52_Routing}

%   depois começa o routing:
After placement, routing is the process that connects every component to each other, and at the PCB level, this can either occur with traces, vias or copper pours.

\subsubsection{Board Setup}\label{sec:521_Board_Setup}

%1. falar das minimum design rules da eurocircuits:
Before beginning the routing process, the board setup must be done in the KiCad PCB editor. This critical step consists in defining the board's stack-up (i.e. number of copper layers -- or simply ``layers'' --, copper pour thickness) and design rules. Depending on the chosen PCB manufacturer, these definitions may vary. For this project, the selected board manufacturer was Eurocircuits, a ``specialist manufacturer and assembler of prototype and small series PCBs''\footnote[19]{Available at \url{https://www.eurocircuits.com/who-are-we/}.}. For KiCad users, Eurocircuists provides sets of minimum design rules (i.e. design constraints) for various PCB stack-up types that may vary in either single or double-sided designs, number of layers, or copper thickness. The following set of constraints presented are for a four-layer board with a base copper thickness of $18 \mu$m and $35 \mu$m, for the outer (OL) and inner (IL) layers, respectively\footnote[20]{Eurocircuits KiCad design rules for other board stack-ups are available at \url{https://www.eurocircuits.com/blog/kicad-design-rules/}.}:
\begin{itemize}
	\item Min. trace width, OL: 0.150mm;
	
	\item Min. clearance, OL: 0.150mm;
	
	\item Min. trace width, IL: 0.150mm;
	
	\item Min. clearance, IL: 0.150mm;

	\item Min. via drill diameter (tool size): 0.35mm;
	
	\item Min. via pad diameter, OL: 0.600mm;
	
	\item Min. via pad diameter, IL: 0.600mm.
\end{itemize}
Therefore, the board's stack-up defined by these constraints was selected as the stack-up for the new beRTK\textsuperscript{\textregistered}'s board. The four-layer order chosen (from front to back) was signal-power-power-signal (in this case: F.Cu-GND-PWR-B.Cu; F.Cu and B.Cu refer to the board's front and back copper signal layers).

A second Eurocircuits' tool can be used to obtain the board's physical stack-up dimensions -- the ``Buildup Editor''. This tool allows the user to select the board's desired number of layers, along with its thickness and base material (among other specifications), and provides a preview of its layers' stack-up with the expected physical dimensions in millimetres. Figure~\ref{fig:buildup_4layer} shows the Eurocircuits Buildup Editor's calculation of the physical stack-up for a four-layer, 1.55mm thick PCB with an FR-4 base material.

\begin{figure}[h]
	\centering
	\includegraphics[width=0.8\textwidth]{Chapters/Figures/chapter5/buildup_4layer.png}
	\caption{Eurocircuits Buildup Editor's calculation of the physical stack-up of a four-layer, 1.55mm thick PCB with an FR-4 Improved base material (available in the Eurocircuits website).}
	\label{fig:buildup_4layer}
\end{figure}

\noindent The values displayed in the diagram of Figure~\ref{fig:buildup_4layer} can thus be applied to the physical stack-up tab of KiCad's board setup manager, as shown in Figure~\ref{fig:KiCad_buildup_4layer}.

\begin{figure}[h]
	\centering
	\includegraphics[width=0.8\textwidth]{Chapters/Figures/chapter5/KiCad_buildup_4layer.png}
	\caption{Board's physical stack-up dimensions applied in KiCad's board setup manager.}
	\label{fig:KiCad_buildup_4layer}
\end{figure}

After setting up the physical stack-up of the board and defining its design rules' constraints, the net classes must be defined. For specified nets (e.g. VBAT, PWR\_LED, GPS\_VDD, etc.), these dictate the actual PCB trace's width and clearance, as well as the size of vias to use. To define a net's trace dimensions, key parameters such as its current flow capacity (in A) temperature rise above ambient ($\Delta T$, in $\degree$C), and copper resistivity ($1.72 \cdot 10^{-8} \Omega$m) must be taken into account. KiCad also provides a PCB calculator tool for trace width calculation, shown in Figure~\ref{fig:PCB_calculator}.

\begin{figure}[h]
	\centering
	\includegraphics[width=0.8\textwidth]{Chapters/Figures/chapter5/PCB_calculator.png}
	\caption{KiCad's PCB track width calculator used.}
	\label{fig:PCB_calculator}
\end{figure}

Looking at Figure~\ref{fig:PCB_calculator}, it is possible to see that the calculator accepts either parameter or layer trace dimension inputs. As stated in the figure, the IPC 2221 (the Generic Standard on Printed Board Design) formula to calculate the trace's maximum current allowed is given by (\ref{eq:I_trace}):

\begin{equation}\label{eq:I_trace}
	I = K \cdot \Delta T^{0.44} \cdot (W \cdot H)^{0.725}\,\medskip
\end{equation}
\noindent Where:
\begin{itemize}
	\item $I$ -- Maximum current that will flow through the trace, in A;
	
	\item $\Delta T$ -- Temperature rise above ambient, in $\degree$C;
	
	\item $W$ -- Trace width, in mils (1 mils $=$ 0.0254 millimeters);
	
	\item $H$ -- Trace thickness (i.e. height), in mils;
	
	\item $K$ -- 0.024 for internal traces or 0.048 for external traces.
\end{itemize}

For this project, four net classes were defined for the routing process:

\begin{itemize}
	\item Default net class -- Corresponds to every net in the circuit that is not HDMI, USB, or power-related. Examples for such are the nets that connect each of the four outputs of comparator U6 to pin header F4, for the status voltage (Figure~\ref{fig:HMI_circuit});
	
	\item Power net class -- Corresponds to every net in the circuit where considerable amount of current is expected to flow. Examples for such are the PWR\_LTC and 5V nets;
	
	\item USB net class -- Corresponds to every USB-related net in the circuit;
	
	\item HDMI net class -- Corresponds to every HDMI-related net in the circuit.
\end{itemize}

It must be noted that, for this project, routing is only done on the outer (i.e. signal) layers of the board. The inner layers are defined in zones. The GND layer is a single, continuous layer that serves as a ground reference for every component, IC, and module; the PWR layer is separated into different zones, to account for dissipation of the circuit's power needs, i.e. to avoid concentrating large amounts of power solely on the outer layers' traces.

%2. falar das minimum design rules que eu usei, que sao um pouco diferentes
%3. dizer a thickness da placa, 4 layers -- 2 signal, 2 pwr -- mostrar a janela do kicad
%4. falar das thickness das tracks de power, de signal, tamanho das vias de power, de signal. clearance
%4.1 definiram se os signal, power, GND planes.

Starting by defining the ``Default'' net class: choosing a trace width of 0.20mm, a trace thickness of $18 \mu$m (as defined earlier in the board stack-up manager), and a common $\Delta T$ temperature rise of $20 \degree$C, plugging those values into KiCad's PCB calculator, a maximum current of approximately 0.624A is allowed to flow through the trace (see Figure~\ref{fig:PCB_calculator}) before any type of trace overheat or breakdown occurs. Requirement \textbf{RTKBS.MAIN.PWS.040} states that the base station ``shall not exceed an average of 400mA of current consumption at 5VDC voltage level''. If every ``Default'' net at 5V consumes the 400mA maximum stipulated, the 0.20mm trace thickness for these nets will be large enough, since it can withstand up to 624mA of current. For this net class, vias were kept at the minimum size allowed by design rule constraints.

The ``Power'' net class encompasses the routing for power-hungry devices in the system, and therefore it must account for larger amounts of current that may flow. Even though the entire system is expected to consume a low amount of power, the Power net class traces and vias should be larger when compared to the Default net class, in case large surges of current or unexpected temperature rises occur. A trace width of 0.50mm was chosen, which allows a maximum current flow of approximately 1.212A. Vias for this net class were defined at a total diameter of 0.80mm with a finished hole size of 0.40mm.

Defining the correct trace dimensions for high-speed circuits is more critical than for default nets or even power nets, for that matter. In design terms, high-speed traces are known as ``microstrips'', since these traces' design allows them to act as transmission lines. Therefore, high-speed circuits are specially sensitive to problems such as signal reflection, coupling, and crosstalk, if designed poorly. The intrinsic inductive and resistive nature of simple traces is often overlooked for typical low-speed circuitry. The same occurs for the capacitive effect between traces -- which calls back to the ``clearance between traces'' parameter. However, for high-speed circuits, these characteristics must never be dismissed, and defining the microstrips is known as ``impedance matching''. It is also critical for these impedance-controlled traces to length-matched, and also referenced to ground -- therefore, the routing of high-speed circuitry must only be done on the top signal layer (F.Cu).

Referring to Section 7.1.1.3 of the USB 2.0 Specification, it defines that microstrips for USB 2.0 data differential pair must bear a nominal differential characteristic impedance of $90 \Omega \pm 15\%$, which means that this impedance value may range from $[76.5,\, 103.5]\Omega$.
The first step in calculating the differential impedance ($Z_{DIFF}$) of a USB 2.0 differential pair of microstrips is to know the single-ended impedance ($Z_0$) value, which can be calculated through expression (\ref{eq:USB_impedance_Z_0})~\cite{USB_Routing}:

\begin{equation}\label{eq:USB_impedance_Z_0}
	Z_0 = \frac{87}{\sqrt{\epsilon_r + 1.41}} \cdot \ln \left(\frac{5.98 \cdot h}{0.8 \cdot w + t}\right)\,\medskip
\end{equation}
$\epsilon_r$ is the substrate's relative permittivity (i.e. dielectric constant) and, for the FR-4 material, is typically equal to $4.5$.
Figure~\ref{fig:USB_differential_pair} shows the cross-section of a PCB with a coupled microstrip line. The traces' and board's dimensions highlighted by the variables are some of the parameters used for the calculation of $Z_0$ and are known as:
\begin{itemize}
	\item $w$ -- Microstrip width, in meters;
	
	\item $d$ -- Microstrip clearance, in meters;
	
	\item $t$ -- Microstrip thickness, in meters;
	
	\item $h$ -- Dielectric thickness, in meters.
\end{itemize}

% meter aqui Microstrips e board diagram
\begin{figure}[h]
    \centering
    \includegraphics[width=0.5\textwidth]{Chapters/Figures/chapter5/USB_differential_pair.pdf}
    \caption{Cross-section of a PCB with a coupled microstrip line.}
    \label{fig:USB_differential_pair}
\end{figure}

Once the $Z_0$ single-ended impedance has been determined, expression (\ref{eq:USB_impedance_Z_DIFF})~\cite{USB_Routing} allows calculating the $Z_{DIFF}$ differential impedance of a coupled microstrip line:

\begin{equation}\label{eq:USB_impedance_Z_DIFF}
	Z_{DIFF} = 2 \cdot Z_0 \cdot \left(1 - 0.48 \cdot e^{-0.96 \cdot \frac{d}{h}}\right)\,\medskip
\end{equation}

\noindent The microstrip width and clearance are the two variables that must be chosen in order to obtain both impedance values, since both $t$ and $h$ thickness values are already fixed at $18 \mu$m and $0.36$mm, respectively. Selecting for the ``USB'' net class the same trace clearance value used in the Default and Power net classes, i.e. $d=0.15$mm, the next step is to choose an adequate microstrip width value that results in a differential impedance value within the $[76.5,\, 103.5]\Omega$ interval ($90\Omega \pm 15\%$). After a few tries, a microstrip width of $w=0.34$mm was chosen for the USB differential pair, which corresponds to a differential impedance of approximately $97.315 \Omega$.

Similarly to the ``USB'' net class, the ``HDMI'' net class is also impedance-controlled and length-matched, since it corresponds to another high-speed circuit. Each HDMI connector features four differential TMDS (Transition-Minimized Differential Signalling) signal pairs, as per~\cite{HDMI_Routing}. These coupled microstrip lines' differential impedance must be kept within $[85,\, 115]\Omega$, i.e. $100\Omega \pm 15\%$. Applying expressions (\ref{eq:USB_impedance_Z_0}) and (\ref{eq:USB_impedance_Z_DIFF}), once again, a differential impedance of approximately $101.509 \Omega$ was obtained for a microstrip width of $0.31$mm, with the same $d=0.15$mm clearance.


%ssSssSssSssSssSssSssSssSssSssSssSssSssSssSssSssSssSssSssSssSssSssSssSssS
\subsubsection{Control Unit and USB 2.0 Hub}\label{sec:521_CM4_USB}

%5. dizer que o primeiro routing que se fez foi o mais critico --  USB e HDMI -- que tambem têm tracks com tamanhos especificos -- falar e mostrar as contas desses tamanhos (impedance matching).
%4. importante que as power tracks passem pelos capacitors primeiro e so depois vao para o power plane (5V), para que os condensadores possam cumprir devidamente o seu propósito de bypassing.

Due to the critical nature of the high-speed parts of the system, the routing process should start at those circuits, which is analogous to what was done in the placement phase.

\begin{figure}[h]
	\centering
	\includegraphics[width=1.0\textwidth]{Chapters/Figures/chapter5/routing_CM4_LAN9514_FCu_BCu.png}
	\caption{Final routing on the front (left) and back (right) signal layers for the CM4 and LAN9514 circuits.}
	\label{fig:routing_CM4_LAN9514_FCu_BCu}
\end{figure}

Figure~\ref{fig:routing_CM4_LAN9514_FCu_BCu} shows the final routing on the front and back signal layers for the CM4 and LAN9514 circuits' components. Looking at the right-hand side of the layout for the front signal layer, the USB 2.0 microstrips can be seen connecting the hub to each downstream port -- it should be reminded that the four ports are represented in the form of two double-stacked ports. Connecting the hub to the CM4's high-speed connector (on the left-hand side of the same layer) is another coupled microstrip line. This differential pair establishes the previously mentioned USB upstream connection. In this same pair, note the skew in the upper trace. Two other similar skews can be seen on the downstream ports' footprints. These ``serpentine tracings'' are used for tuning the microstrip's length, in order for it to match, as close as possible, the length of its counterpart. Figure~\ref{fig:USB_zoom} shows a detailed view of the high-speed routing containing these length-tuning skews.

\begin{figure}[h]
	\centering
	\includegraphics[width=0.8\textwidth]{Chapters/Figures/chapter5/USB_zoom.png}
	\caption{Details of the high-speed routing on the LAN9514 circuit.}
	\label{fig:USB_zoom}
\end{figure}

%se for preciso, escrever sobre matching lengths

Besides the high-speed connections mentioned, every other trace visible belongs to the ``Default'' net class, apart from the visibly wider ones, which are part of the ``Power'' net class (see Figure~\ref{fig:routing_CM4_LAN9514_FCu_BCu}). For the latter, an exception was made for the 3V3 net from the CM4. The CM4 pins that output 3.3V meet and merge into a $1.50$mm-wide trace, in order to account for large current surges and to ease power dissipation (visible on the front layer routing represented also in Figure~\ref{fig:routing_CM4_LAN9514_FCu_BCu}).


%ssSssSssSssSssSssSssSssSssSssSssSssSssSssSssSssSssSssSssSssSssSssSssSssS
\subsubsection{Power Selector and Battery Balancer}\label{sec:522_LTC4012_BQ29209}

Following the layout priority order suggested by~\cite{LTC4012} --  as mentioned in Section~\ref{sec:5112_LTC4012_BQ29209} --, the routing process for the power selector started on its switch node. This is a critical node where FET switching takes place, at a high-frequency. Current values and dissipated power can easily be higher than usual on this node, and therefore it was classified as a ``Power'' net. However, to account for larger current surges and higher power dissipation, the connections between switching FETs (Q2 and Q9) to diode D4 and inductor L1, and also the latter to sense resistor R12 were routed as a $1.50$mm-wide trace (similarly to what was done to the 3V3 power net). This type of tracing is also done on the DCIN node (i.e. from the external adapter input to input PFET controller Q1 -- see Figure~\ref{fig:LTC4012_circuit}) and then across sense resistor R3 to the PWR\_LTC net.

The remaining connections are made in a standard way, as short as possible according to the placement, and only relying on vias when absolutely needed. The final routing of the power selector, on the front and back signal layers of the board, is shown on the right-hand side of Figure~\ref{fig:2_routing_LTC4012_BQ29209_FCu_BCu}.

%2_routing_LTC4012_BQ29209_FCu_BCu
\begin{figure}[h]
	\centering
	\includegraphics[width=0.7\textwidth]{Chapters/Figures/chapter5/2_routing_LTC4012_BQ29209_FCu_BCu.png}
	\caption{Final routing on the front (left) and back (right) signal layers for the Power Selector and Battery Balancer circuits.}
	\label{fig:2_routing_LTC4012_BQ29209_FCu_BCu}
\end{figure}

At the left-hand side of Figure~\ref{fig:2_routing_LTC4012_BQ29209_FCu_BCu} the battery balancer can be seen. For its power nets, the traces' width is the defined $0.50$mm, and similarly to the routing in the LTC4012's circuit, a particular set of nets have wider traces. In this case, a $1.0$mm-wide trace is used to connect the battery pack connector's (F9) terminals to the circuit, in order to deliver the needed power in the safest way possible. And as is other wide-trace nets, this width also aids in current flow and power dissipation.


%ssSssSssSssSssSssSssSssSssSssSssSssSssSssSssSssSssSssSssSssSssSssSssSssS
\subsubsection{Voltage Converter}\label{sec:524_VoltageConverter}

For the 5V supply, i.e. the AP64501,~\cite{AP64501} suggests a layout design whose placement was covered earlier, in Section~\ref{sec:5114_VoltageConverter}.This layout suggestion is featured in Figure~\ref{fig:AP64501_layout_Datasheet}.
Since the base station's system is designed to be low-power, high amounts of current are not expected to be drawn from the power supplies, and therefore the layout part visible on Figure~\ref{fig:AP64501_layout_Datasheet} is not deemed necessary, since wider traces and isolated zones on the PCB's PWR layer are able to provide trustworthy routing, as shown on Figure~\ref{fig:3_routing_Voltage_Converter_FCu}.

%3_routing_Voltage_Converter_FCu
\begin{figure}[h]
	\centering
	\includegraphics[width=0.6\textwidth]{Chapters/Figures/chapter5/3_routing_Voltage_Converter_FCu.png}
	\caption{Final routing on the front signal layer for the Voltage Converter circuit.}
	\label{fig:3_routing_Voltage_Converter_FCu}
\end{figure}

It should be noted that special care was taken when routing the 5V output net (from inductor L2). To provide the desired voltage with the lowest amount of disturbances possible, a $1.50$mm-wide trace is routed from terminal 2 of the L2 inductor and, before reaching the VCC (+5V) terminal -- which in turn is connected to an isolated zone in the board's PWR layer for the 5V net --, it first must run across the three output capacitors, C40, C42 and C43 (see Figure~\ref{fig:AP64501_circuit} for reference). This way these capacitors can properly fulfil their bypassing purpose for this power supply.


%ssSssSssSssSssSssSssSssSssSssSssSssSssSssSssSssSssSssSssSssSssSssSssSssS
\subsubsection{Power Switch}\label{sec:525_PowerSwitch}

Just like the placement of the power switch circuit, its routing process was also simple, with special attention devoted to the higher-loaded connections, such as between the PWR\_LTC and the PWR\_AP64501 nets, i.e. across FET Q4 (see Figure~\ref{fig:SWITCH_circuit}). Figure~\ref{fig:4_routing_Switch_FCu_BCu} shows the final routing on the front and back layers of the board for this circuit, where these latter connections can be observed.

%4_routing_Switch_FCu_BCu
\begin{figure}[h]
	\centering
	\includegraphics[width=0.8\textwidth]{Chapters/Figures/chapter5/4_routing_Switch_FCu_BCu.png}
	\caption{Final routing on the front (left) and back (right) signal layers for the Power Switch circuit.}
	\label{fig:4_routing_Switch_FCu_BCu}
\end{figure}


%ssSssSssSssSssSssSssSssSssSssSssSssSssSssSssSssSssSssSssSssSssSssSssSssS
\subsubsection{Human-Machine Interface}\label{sec:526_HMI}

Routing the LM2901 comparator's circuit was analogous to its placement -- straightforward.

It must be noted that, since this IC is used to compare the battery pack's voltage with a voltage reference (U7), all nets that connect those ICs to the ``status resistors'' (R24, R29, R30, R31, and R32; see Figure~\ref{fig:HMI_circuit}) belong to the ``Power'' net class, as it is possible to verify though Figure~\ref{fig:5_routing_HMI_FCu_1}.

%5_routing_HMI_FCu_1
\begin{figure}[h]
	\centering
	\includegraphics[width=0.8\textwidth]{Chapters/Figures/chapter5/5_routing_HMI_FCu_1.png}
	\caption{Final routing on the front (left) and back (right) signal layers for the status voltage comparator circuit.}
	\label{fig:5_routing_HMI_FCu_1}
\end{figure}

As previously mentioned, the individual HMI PCB is adjacent to the main system's board, and its routing just consists in connecting push-button SW1 and all LEDs used for power-status indication to pin header F5. Figure~\ref{fig:5_routing_HMI_FCu_2} presents said routing.

%5_routing_HMI_FCu_2
\begin{figure}[h]
	\centering
	\includegraphics[width=0.8\textwidth]{Chapters/Figures/chapter5/5_routing_HMI_FCu_2.png}
	\caption{Final routing on the front (left) and back (right) signal layers for the HMI circuit.}
	\label{fig:5_routing_HMI_FCu_2}
\end{figure}


%ssSssSssSssSssSssSssSssSssSssSssSssSssSssSssSssSssSssSssSssSssSssSssSssS
\subsubsection{GNSS module and Wi-Fi Transceiver}\label{sec:527_ZED_XBEE}

For the GNSS module, the routing's most important section corresponds to the module's power input, which is connected to the GPS\_VDD power net. As done in the routing of the 5V voltage converter, before this trace reaches the power input for the GNSS module, it must first connect to the input bypass capacitors C9 and C10 (see Figure~\ref{fig:ZEDF9P_circuit}), to filter out signal abnormalities.

As for the Wi-Fi transceiver, the important routing section also refers to the power connections. In this case, after the step-down conversion from 5V to 3.3V by LDO U2, the trace delivering these 3.3V to the XBee module must first run across bypass capacitors C11-C14.

For both modules, the remaining routing process was standard. Figure~\ref{fig:6_routing_ZED_XBee} shows the routing of both modules' circuits.

%6_routing_ZED_XBee
\begin{figure}[h]
	\centering
	\includegraphics[width=0.8\textwidth]{Chapters/Figures/chapter5/6_routing_ZED_XBee.png}
	\caption{Final routing on the front (left) and back (right) signal layers for the GNSS module and Wi-Fi Transceiver circuits.}
	\label{fig:6_routing_ZED_XBee}
\end{figure}

%agora falar da full board:
While routing the entire board, the copper pours for all four layers -- which were already defined as the board's geometric shape -- were repeatedly updated (i.e. ``filled''). Special care was also taken for both signal layers by adding evenly-spaced free-standing vias from these layers to the GND layer, in order to define a solid and reliable ground reference plane. All this was done in order to avoid re-routing as much as possible due to, for example, the hypothetical existence of numerous copper islands\footnote[21]{Small, unconnected zones of copper on a PCB layer. In high-speed circuits, these zones can act as small antennas, causing problems such as signal reflection, or crosstalk. In case removal is not practical/possible, these islands should be connected to the GND layer using isolated vias.}, among other issues. Figure~\ref{fig:7_routing_FULL_FCu_BCu} shows a general view of the final routing on the front and back signal layers, for the entire board.

%7_routing_FULL_FCu_BCu
\begin{figure}[h]
	\centering
	\includegraphics[width=0.8\textwidth]{Chapters/Figures/chapter5/7_routing_FULL_FCu_BCu.png}
	\caption{Final routing on the front (left) and back (right) signal layers for the new beRTK\textsuperscript{\textregistered} system's circuit.}
	\label{fig:7_routing_FULL_FCu_BCu}
\end{figure}

Regarding the PWR layer, it was the only manipulated layer (in terms of geometry), since it was subdivided into eight different zones for the main board, corresponding to the system's power nets considered most important (due to current flow, power dissipation, etc.):

\begin{itemize}
	\item 5V;
	\item 3V3;
	\item VDD33A;
	\item PWR\_AP64501;
	\item PWR\_LTC;
	\item VBAT;
	\item DCIN;
	\item SW (LTC4012).
\end{itemize}

On the HMI adjacent board, this layer instead acts as a ground reference layer -- i.e. it essentially performs as another GND layer --, since this board does not need a PWR layer for power purposes. 

Each of the PWR zones on the main is evenly-spaced from each other by $2.0$mm gaps. This distance guarantees electrical isolation between the separated PWR zones, which is critical in order to avoid dielectric breakdown. Figure~\ref{fig:dielectric_breakdown} represents a theoretical top view of two hypothetical zones on the PWR layer separated by the board's dielectric, known as FR-4.

% meter aqui dielectric_breakdown:
\begin{figure}[h]
    \centering
    \includegraphics[width=0.5\textwidth]{Chapters/Figures/chapter5/dielectric_breakdown.pdf}
    \caption{Dielectric breakdown on the PWR layer.}
    \label{fig:dielectric_breakdown}
\end{figure}

Looking at Figure~\ref{fig:dielectric_breakdown}, three variables to define the dielectric breakdown can be seen, and are defined as such:
\begin{itemize}
	\item $g$ -- Dielectric gap between copper zones. Selected as $2.0$mm;
	
	\item $d_s$ -- Dielectric strength for the FR-4 material. Equal to $20 \cdot 10^{6}$V/m;
	
	\item $\epsilon_r$ -- Dielectric constant for the FR-4 material. Typically equal to $4.5$.
\end{itemize}

The dielectric strength of $d_s=20 \cdot 10^{6}$V/m means the FR-4 dielectric between the PWR zones will break down if an electric field stronger than $20 \cdot 10^{6}$V/m is applied between them. Thus, in order to verify if the $2.0$mm gap is large enough to avoid dielectric break down, the maximum voltage that can be applied between the copper layers must be determined. Expression (\ref{eq:dielectric_breakdown}) can be used for such:

\begin{equation}\label{eq:dielectric_breakdown}
	V_{max} = d_s \cdot g = 20 \cdot 10^{6} \cdot 2 \cdot 10^{-3} = 40,000 \textrm{V}\,\medskip
\end{equation}

Therefore, selecting a gap of $2.0$mm provides a more-than-enough isolation between each PWR layer, since the maximum voltage that can be applied before the FR-4 dielectric breaks down is orders of magnitude above the maximum voltage that may be applied to the system (22V).

Figure~\ref{fig:8_FCu_GND_PWR_BCu} presents all the copper pours for every layer (with all subdivisions of the PWR layer respectively labelled).

%8_FCu_GND_PWR_BCu
\begin{figure}[h]
	\centering
	\includegraphics[width=0.6\textwidth]{Chapters/Figures/chapter5/8_FCu_GND_PWR_BCu.png}
	\caption{New beRTK\textsuperscript{\textregistered} system board's copper pour for (from top-down, left-right): front signal layer (F.Cu); ground layer (GND); power layer (PWR); back signal layer (B.Cu).}
	\label{fig:8_FCu_GND_PWR_BCu}
\end{figure}


%ssSssSssSssSssSssSssSssSssSssSssSssSssSssSssSssSssSssSssSssSssSssSssSssS



-- $20 \mu$F ceramic capacitors were chosen for both the power inputs and power output of the LTC4012, instead of $22 \mu$F. 

-- As mentioned before, the top and bottom power FETs Q2 and Q9, along with inductor L1 are vital to the PWM control architecture. These components are part of the sub-circuit that starts from the +15VDC source of power and passes across FET Q1 (connected to pin INFET), sense resistor R3, 
This sub-circuit forms the ``power supply rail'', and upon layout design of the circuit from Figure~\ref{fig:LTC4012_circuit}, this section must bear a track width large enough to withstand large values of currents or any other phenomena that may occur (e.g. voltage/current spikes). The needed track width can be calculated through the following expression:

XBee: Digi XBee 3 RF Module Hardware Reference Manual
We design XBee 3 RF Modules to be self-sufficient and have minimal sensitivity to nearby processors,
crystals or other printed circuit board (PCB) components. Keep power and ground traces thicker than
signal traces and make sure that they are able to comfortably support the maximum current
specifications. There are no other special PCB design considerations to integrate XBee 3 RF Modules,
with the exception of antennas.

\section{PCB Manufacture}\label{sec:51_PCBmanufacture}

\section{Component Acquisition}\label{sec:52_ComponentAcquisition}

%ate agora foram os unicos MOSFETs que nao disse o nome:

%como nao foi relevante, ate agora ainda nao disse que MOSFETs selecionei para o power switch: Q4 e Q5.
%Q3, Q10, Q6, Q8 -- BQ29209
%PMEG2010ER - GNSS module

%LTC4012:
It should be noted that, for the top and bottom FET selection stage, the Si7212DN model suggested is a double FET in one package. The CSD17308Q3 substituting model is a single-channel FET (only one per package), and therefore two units had to be acquired -- Section~\ref{sec:3211_LTC4012}.

% AP64501:
Recalling Section~\ref{sec:3214_AP64501}, it is stated that the closest commercially available values of 11k$\Omega$ and 2.7k$\Omega$ were selected for R10 and R13, respectively, at the prototyping phase. Following expressions (\ref{eq:R39}) and (\ref{eq:R40}), these values allow the expected $V_{ON}$ and $V_{OFF}$ values.
    % inductor L2:
    Meter Eq. 9 do datasheet do AP64501;
    
    Peak current determines the required saturation current rating, which influences the size of the inductor. Saturating the inductor decreases the converter efficiency while increasing the temperatures of the inductor and the internal power MOSFETs. Therefore, choosing an inductor with the appropriate saturation current rating is important. 
    It is recommended by \cite{AP64501} the selection of an inductor value between $1 \mu$H to $10 \mu$H, and therefore, taking into account the values presented in Table~\ref{tab:AP64501_recommended_values}, an inductor of $3.6 \mu$H was selected. It is also advised to select an indcutor with a DC current rating of at least 35\% higher than the maximum 5A load current of the AP64501, which corresponds to 6.75A.
    For highest efficiency, the inductor's DC resistance should be less than 10mOhm. Use a larger inductance for improved efficiency under light load conditions.


% CM4:
    At just 40mm $\times$ 55mm and 4.7mm deep, the form factor of the CM4 is one of its biggest advantages, since it allows designers to fit it in almost all designs.

    %HDMI routing:
    HDMI signals should be routed as 100Ohm differential pairs. Each signal within a pair should ideally be matched to better than 0.15mm. Pairs don't typically need any extra matching, as they only have to be matched to 25mm.

    %USB routing:
    The differential pair should be routed as a 90Ohm differential pair. The length of the P/N signals should ideally be matched to better than 0.15mm.

    6. The port is capable of being used as a true USB On-The-Go (OTG) port. While there is no official documentation, some users have had success making this work. The USB\_OTG\_ID pin is used to select between USB host and device that is typically wired to the ID pin of a Micro USB connector. To use this functionality it must be enabled in the OS. If using either as a fixed slave or fixed master, please tie the USB\_OTG\_ID pin to ground. tentei isto e nao deu.

	-- It also states that, in order to assemble this module onto a carrier board, one of two possible mating connectors must be chosen. The two options are named DF40C-100DS-0.4v and DF40HC(3.0)-100DS-0.4v and when installed, provide a clearance under the CM4 of 0mm and 1.5mm, respectively. 

% LAN9514:
    However, USB has the advantage of allowing hot-swapping, making it useful for mobile peripherals, including drives of various kinds. -- mesmo que digam isto ja é possivel por natureza do USB, eu ainda tentei fazer o que dizia no datahseet do LAN9514 para dar enable ao hot-swapping, mas continuou sem funcionar.

\section{Prototype Assembly}\label{sec:53_PrototypeAssembly}

\section{Functional Testing and Results}\label{sec:54_FunctionalTesting}